%!TEX root = ../var.tex
\textbf{Пример}. Игральную кость подбрасывают один раз. Известно, что выпало более трёх очков. Какова при этом вероятность того, что выпало чётное число очков?

\textit{Решение}. $\Omega_1 = {4, 5, 6}, A = {4, 6}, и P(A) = \mu(A)
\mu(\Omega) = 3$ .

\begin{zam}
Пусть $\Omega$ -- пространство элементарных событий и $B \subset \Omega$ -- событие, отличное от невозможного, т.е. $B \ne \noo$. 
Пусть $A \subset \Omega$ – другое событие. Какова вероятность того, что произойдёт событие $A$, при условии,
что событие $B$ произошло? Слова \textit{событие $B$ произошло означают}, что новым пространством элементарных событий становится событие $\Omega_1 = B$ и его мера равна $\mu(\Omega_1 ) = \mu(B)$. 

Слова \textit{произойдёт событие $A$, если событие $B$
произошло} означают ту часть события $A$, которая содержится в $B$, т.е. означают \textit{произойдёт событие} $A \cap B$. Ясно, что $A \cap B \subset \Omega$ и $A \cap B \subset B = \Omega_1$.
\end{zam}

\begin{definition}
Для того, чтобы подчеркнуть, что событие $A \cap B$ есть
событие из нового пространства элементарных событий $\Omega_1 = B$ его обозначают $A|B$ и называют \textit{событие $A$ при условии, что событие $B$ произошло.}

Очевидно, что $P(A|B)=\frac{\mu(A|B)}{\mu(B)}$. Вероятность $P(A|B)$ называется \textit{условной вероятностью}.
\end{definition}

\begin{lemma}
	$P(A|B) = \frac{P(A\cap B)}{P(B)}$.
\end{lemma}

\begin{proof}
\begin{equation*}
	P(A|B) = \frac{\mu(A|B)}{\mu(B)}=\frac{\mu(A\cap B)}{\mu(B)}
=\frac{\frac{\mu(A\cap B)}{\mu(\Omega)}}{\frac{\mu(B)}{\mu(\Omega)}}
=\frac{P(A\cap B)}{P(B)}.
\end{equation*}	
\end{proof}

Следующая формула непосредственно следует из леммы 6.3 и традиционно называется \textit{теоремой умножения}.

\begin{theorem}[Теорема умножения для двух событий]
Если $P(B) > 0$, $P(A) > 0$, то
\begin{equation*}
	P(A \cap B) = P(B)P(A|B) = P(A)P(B|A).
\end{equation*}
\end{theorem}

\begin{theorem}[Теорема умножения для n событий]
\begin{equation*}
	P(A_1 \cap A_2 \cap\ldots\cap A_n ) = P(A_1)P(A_2 |A_1 )P(A_3 |A_1 \cap A2 )\cdot... \cdot P(A_n |A_1 \cap A_2 \cap\ldots\cap A_{n−1}),
\end{equation*}
если все условные вероятности определены.
\end{theorem}
\begin{proof}
Доказать методом математической индукции.	
\end{proof}
