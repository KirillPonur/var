%!TEX root = ../var.tex

Непосредственным обобщением схемы независимых испытаний (см. опред.
10.3) является схема цепей Маркова.

\begin{definition}
Пусть эксперимент состоит в том, что
\begin{enumerate}
	\item проводят последовательность испытаний с номерами $s = 1, 2, \ldots$ ;\\
	\item в каждом испытании появляется одно из $k$ несовместных событий
	$$A_1 , A_2 , \ldots , A_k,$$ при этом то, что в $s$-ом испытании произошло событие $A_i$ , будем обозначать $A^s_i$ ;\\
	\item вероятность появления события $A^{s+1}_j$ (то есть $j$-го события в $(s+1)$-ом испытании) зависит только от того, какое событие произошло в предыдущем $s$-ом испытании и не зависит от того, какие события произошли в испытаниях с номерами $s − 1, s − 2, \ldots $.
Тогда такая последовательность событий называется (простой) цепью Маркова.
\end{enumerate}

\end{definition}

\textbf{Пример.} Согласно предложенной Н. Бором\footnote{Нильс Хенрик Давид Бор (Niels Henrik David Bohr, 1865 — 1962), датский физик, один из создателей
квантовой физики. Создал первую квантовую модель атома, участвовал в разработке основ квантовой
механики, теории атомного ядра, ядерных реакций и взаимодействия элементарных частиц со средой.
} модели атома водорода электрон может находится только на одной из допустимых орбит. Обозначим через  $A_i$ событие, состоящее в том, что электрон находится на
$i$-ой орбите. Эти события Бор назвал стационарными состояниями атома. Вероятность перехода электрона с $i$-ой на $j$-ю орбиту зависит только от $i$ и $j$, потому что разность $j − i$ зависит от количества испускаемой
или поглощаемой атомом энергии и не зависит от того, на каких орбитах находился электрон в прошлом. Этот пример доставляет цепь Маркова теоретически с бесконечным числом состояний атома.

Теория цепей Маркова довольно обширна, поэтому мы ограничимся изучением так называемых однородных цепей Маркова.

Рассмотрим условную вероятность $\P(A^{s+1}_j|A^s_i)$,т.е. вероятность появления события $A_j^{s+1}$ при условии, что событие $A^s_i$ произошло.

\begin{definition}
Цепь Маркова называется однородной, если условная вероятность $\P(A^{s+1}_j|A^s_i)$ не зависит от номера $s$.

При этом вероятность $\P(A^{s+1}_j|A^s_i)$ называется вероятностью перехода от события $A^s_i$ к событию $A^{s+1}_j$ и обозначается
$$p_{ij} = \P(A^{s+1}_j|A^s_i)$$
\end{definition}

\begin{zam}
Ясно, что

1) $0 \leq p_{ij} \leq 1$ и

2) $p_{i1} + p_{i2} + \ldots + p_{ik} = 1$ для всех $i = 1, 2, \ldots , k.$
\end{zam}

\begin{definition}
Полная информация о вероятностях перехода в цепи Маркова содержится в таблице

$$\pi_1 = \begin{pmatrix} 
p_{11} & p_{12} & \ldots & p_{1k}\\
p_{21} & p_{22} & \ldots & p_{2k}\\
\ldots & \ldots & \ldots & \ldots\\
p_{k1} & p_{k2} & \ldots & p_{kk}\\
\end{pmatrix}
$$
которая называется матрицей перехода.
\end{definition}

\begin{zam}
Главной задачей теории цепей Маркова является нахождение вероятности перехода от события $A^s_i$ к событию $A^{s+n}_
j$ , произошедшему через $n$ испытаний. Обозначим эту вероятность
$$p_{ij}(n) = \P(A^{s+n}_j|A^s_i),$$
а искомую матрицу перехода через n испытаний через
$$\pi_n = 
\begin{pmatrix} 
p_{11}(n) & p_{12}(n) & \ldots & p_{1k}(n)\\
p_{21}(n) & p_{22}(n) & \ldots & p_{2k}(n)\\
\ldots & \ldots & \ldots & \ldots\\
p_{k1}(n) & p_{k2}(n) & \ldots & p_{kk}(n)$$
\end{pmatrix}
$$
\end{zam} 

\begin{theorem}
	Если $0 < m < n$, то $\pi_n = \pi_m \cdot \pi_{n−m} .$
\end{theorem}

\begin{proof}
Рассмотрим какое-нибудь промежуточное испытание с номером $s + m$, то есть $s < s + m < s + n$. Пусть в этом испытании появится какое-то событие $A^{s+m}_r$, где $1 \leq r \leq k$. В наших обозначениях вероятность перехода от события $A^s_i$ к событию $A^{s+m}_r$ равна
$$\P(A^{s+m}_r|A^s_i) = p_{ir}(m),$$
а вероятность перехода от события $A^{s+m}_r$ к событию $A^{s+n}_j$ равна
$$\P(A^{s+n}_j|A^{s+m}_r) = p_{ir}(n-m)$$
По формуле полной вероятности 7.2 имеем
$$p_{ij}(n) = \sum\limits_{r+1}^k p_{ir}(m) \cdot p_{rj}(n-m).$$

Последняя формула в точности совпадает с формулой произведения матриц, поэтому получаем $\pi_n = \pi_m \cdot \pi_{n−m} .$
\end{proof} 

\begin{theorem}
 	Для любого $n \geq 1$, имеет место формула $\pi_n = \pi_1^n$ .
 \end{theorem} 

\begin{proof}
Доказательство проведём методом математической индукции.
\begin{enumerate}
	\item Проверяем тождество при $n = 1$: $\pi_1 = \pi^1_1 = \pi_1$.
	\item Пусть при $n = q$ выполнено тождество $\pi_q = \pi_1^q .$
	\item По теор. 28.6 имеем $\pi_{q+1} = \pi_1 \cdot \pi_q = \pi_1 \cdot \pi_1^q = \pi_1^{q+1}$.
\end{enumerate}
Что и требовалось доказать.
\end{proof}