%!TEX root = ../var.tex
Напомним, что мы рассматриваем случайные величины, имеющие конечные начальные и центральные моменты до $n$-го порядка включительно.
Метод доказательства неравенств, изучаемых в этом параграфе, принадлежит Чебышёву\footnote{
Чебышёв, Пафнутий Львович (распространено неправильное произношение его фамилии с ударением на первый слог) (1821 — 1894), выдающийся русский математик и механик, внёсший большой вклад
в теорию вероятностей, теорию приближений, теорию интерполирования функций, интегральное исчисление и картографию. Работая на <<оборонку>>, он улучшил дальнобойность и точность артиллерийской
стрельбы, чем оказал большое влияние на развитие русской артиллерии.	
}.

\begin{theorem}[Неравенство Маркова\footnote{
	Марков, Андрей Андреевич (1856 — 1922), выдающийся русский математик, внёсший большой вклад
в теорию вероятностей и математический анализ.
} , 1913 г.]
\label{th:20.1}
 Для любой случайной
величины $\xi$ и для любого $\epsilon \in (0, \infty)$, имеет место неравенство

\begin{equation*}
\P( |\xi| \geq \epsilon ) \leq \frac{\M|\xi|}{\epsilon}	
\end{equation*}	
\end{theorem}

\begin{proof}
	Если $c$ -- плотность случайной величины $\xi$, то

\begin{equation*}
	\P( |\xi| \geq \epsilon ) = \int\limits_{|x|\geq \epsilon} f_{\xi}(t)dt
\end{equation*}

Ясно, что для всех $t \in \{ |x| \geq \epsilon \}$ выполнено неравенство $\frac{|t|}{\epsilon} \geq 1$. Поэтому при замене $f_{\xi}(t)$ на $\frac{|t|}{\epsilon} f_{\xi}(t)$ подынтегральное выражение не уменьшится, т.е.

\begin{equation*}
\P( |\xi| \geq \epsilon ) = \int\limits_{|x| \geq \epsilon}f_{\xi}(t)dt \leq \int\limits_{|x| \geq \epsilon} \frac{|t|}{\epsilon} f_{\xi}(t)dt = \frac{1}{\epsilon}\int\limits_{|x| \geq \epsilon} |t|f_{\xi}(t)dt. 
\end{equation*}

Если теперь мы увеличим область интегрирования с ${ |x| \geq \epsilon }$ до $(−\infty, \infty)$,
то интеграл справа тоже не уменьшится. Окончательно получаем

\begin{equation*}
	\P( |\xi| \geq \epsilon ) \leq \frac{1}{\epsilon} \int\limits_{|x| \geq \epsilon} |t|f_{\xi}(t)dt \leq \frac{1}{\epsilon} \int\limits_{-\infty}^{\infty} |t|f_{\xi}(t)dt = \frac{\M|\xi|}{\epsilon}
\epsilon
\end{equation*}
\end{proof}


\begin{consq}[Двойственное неравенство Маркова]
\label{consq:20.2}
	Для любой случайной величины $\xi$ и для любого $\epsilon \in (0, \infty)$ имеет место неравенство
$$\P( |\xi| < \epsilon ) \geq 1 − \frac{\M|\xi|}{\epsilon}$$
\end{consq}

\begin{proof}
 	По лемме \ref{lemma:3.7} имеем $\P( |\xi| < \epsilon ) = 1 − \P( |\xi| \geq \epsilon ) $. Подставим это выражение в неравенство Маркова, получим требуемый результат.
 \end{proof} 

\begin{theorem}[Обобщённое неравенство Маркова]
\label{th:20.3}
 Для любой случайной величины $\xi$, для любого $\epsilon \in (0, \infty)$ и любой монотонно возрастающей
функции $g : (0, \infty) \to (0, \infty)$ имеет место неравенство
$$\P( |\xi| \geq \epsilon ) \leq \frac{\M g(|\xi|)}{g(\epsilon)}$$
 \end{theorem} 
 
 \begin{proof}
 	Поскольку функция $g$ монотонно возрастает, то
$\P( |\xi| \geq \epsilon ) = \P( g(|\xi|) \geq g(\epsilon) )$. Оценивая последнюю вероятность согласно неравенству Маркова, получим требуемый результат:

\begin{equation*}
	\P(g(|\xi|) \geq g(\epsilon)) \leq \frac{\M g(|\xi|)}{g(\epsilon)}
\end{equation*}
 \end{proof}
 
 \begin{consq}
 \label{consq:20.4}
Для любой случайной величины $\xi$, для любого $\epsilon \in (0, \infty)$, и любой монотонно возрастающей функции $g : (0, \infty) \to (0, \infty)$ имеет место неравенство

\begin{equation*}
	\P(g(|\xi|)<g(\epsilon)) >1- \frac{\M g(|\xi|)}{g(\epsilon)}
\end{equation*}
(двойственное к обобщённому неравенству Маркова).
Доказательство аналогично доказательству следствия \ref{consq:20.2}.
В 1853 г. И.-Ж. Бьенеме\footnote{
	И.-Ж. Бьенеме (Irenee-Jules Biemayme, 1796 — 1878), французский математик, основные работы по теории вероятностей и математической статистике.} 
и в 1866 г. независимо от него П.Л. Чебышёв
доказали следующее неравенство. 	
 \end{consq}

\begin{theorem}[Неравенство Бьенеме – Чебышёва]
\label{th:20.5}
Для любой случайной величины $\xi$ и любого $\epsilon \in (0, \infty)$ имеет место неравенство
$$\P( |\xi − \M\xi| \geq \epsilon ) \leq \frac{\D\xi}{\epsilon^2}$$
\end{theorem}

\begin{proof}
Т.к. неравенство Маркова \ref{th:20.1} справедливо для произвольной случайной величины, то перепишем его для случайной величины $\xi − \M\xi$, получим 

\begin{equation*}
	\P( |\xi − \M\xi| \geq \epsilon ) \leq \frac{\M|\xi - \M\xi|^2}{\epsilon}
\end{equation*}

Зададим функцию $g : (0, \infty) \to (0, \infty)$ по формуле $g(\epsilon) = \epsilon^2$. Видно,
что она является монотонно возрастающей и поэтому удовлетворяет условиям теоремы \ref{th:20.3}, и следовательно, для неё выполнено обобщённое неравенство Маркова
$$\P( |\xi − \M\xi| \geq \epsilon ) \leq \frac{\M(\xi - \M\xi)^2}{\epsilon^2} = \frac{\D\xi}{\epsilon^2} $$
\end{proof}

\begin{consq}
\label{consq:20.6}
Для любой случайной величины $\xi$ и любого $\epsilon \in (0, \infty)$, имеет место неравенство
$$\P( |\xi − \M\xi| < \epsilon ) \geq 1 − \frac{\D\xi}{\epsilon^2}$$
(двойственное к неравенству Бьенеме – Чебышёва).
Мы будем называть двойственное неравенство тоже неравенством Бьенеме – Чебышёва.
\end{consq}

\begin{consq}
\label{consq:20.7}
Для любой случайной величины $\xi$ вероятность того,
что она примет значение, отличающееся от её среднего $\M\xi$ более чем на три корня из её дисперсии, не превосходит $\frac{1}{9}$, т.е.
$$\P(|\xi - \M\xi| \geq 3\sqrt{\D\xi}) \leq \frac{1}{9}$$
\end{consq}

\begin{proof}
Из неравенства Бьенеме-Чебышёва имеем
$$ \P(|\xi − \M\xi| \geq 3\sqrt{\D\xi}) \leq \frac{\D\xi}{(3\sqrt{\D\xi})^2} = \frac{1}{9}$$
 \end{proof} 