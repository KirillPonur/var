%!TEX root=../var.tex
\begin{example}
Пусть случайная величина $\xi$ имеет распределение Бернулли
\begin{center}
\begin{tabular}{|c|c|c|}
\hline
$\xi$ & $0$ & $1$\\ \hline
$P$ & $1-p$ & $p$\\ \hline
\end{tabular}	
\end{center}
По лемме 24.10 её характеристическая функция равна
$$\Theta\xi (t)=\M e^{it\xi}=e^{it\cdot0}(1-p)+e^{it\cdot1}p=1-p+pe^{it}$$
\end{example}

\begin{example}
Пусть случайная величина $\xi$ имеет биномиальное распределение (см. опред. 9.3) $k \mapsto \P(\xi_n=k)=C_n^k p^k (1-p)^{n-k}$ , где 
$k=0, 1, 2, \ldots , n$. Её характеристическая функция равна
%
%
$$\Theta\xi (t)=\M e^{it\xi}=
\sum\limits_{k=0}^n e^{itk} C_n^k p^k (1-p)^{n-k}=
\sum\limits_{k=0}^n C_n^k (pe^{it})^k (1-p)^{n-k}=(1-p+p^{it})^n,
$$
где последнее равенство есть бином Ньютона.
\end{example}


\begin{example}
Пусть случайная величина $\xi$ имеет распределение Пуассона (см. опред. 23.2) $k \mapsto \P\lambda (k)=\frac{\lambda^k}{k!}e^{-\lambda}$, где $k=0, 1, 2, \ldots$. 

По лемме 24.10 её характеристическая функция равна
%
%
$$\Theta\xi (t)=\M e^{it\xi}=
\sum\limits_{k=0}^n e^{itk} \frac{\lambda^k}{k!}e^{-\lambda}=
e^{-\lambda}\sum\limits_{k=0}^n \frac{(\lambda e^{it})^k}{k!}=
e^{-\lambda}e^{\lambda e^{it}}=
e^{\lambda(e^{it}-1)}
$$
\end{example}

\begin{example}
Пусть случайная величина $\xi$ имеет показательное распределение (см. опред. 12.7), 
%
$f_\xi(x)=\left\{
\begin{aligned}
	0&, x&<0
	\lambda e^{-\lambda x}, x&\geq0
\end{aligned}
\right.$.
%
По лемме 24.10 её характеристическая функция равна
%
%
\begin{gather*}
\Theta\xi (t)=\M e^{it\xi}=
\int\limits_0^\infty e^{itx}\lambda e^{-\lambda x} dx=
\lambda\int\limits_0^\infty e^{-(\lambda-it)x} dx=\\=
\frac{\lambda}{\lambda-it}
\left(
	-e^{-(\lambda-it)x}\bigg|_0^\infty
\right)=\frac{\lambda}{\lambda-it},
\end{gather*}


поскольку при $x \to \infty$ модуль величины $e^{-(\lambda-it)x}=e^{itx}\lambda e^{-\lambda x}$ стремится к
нулю: $|e^{-(\lambda-it)x}|=e^{-\lambda x}\to0$.
\end{example}

\begin{example}
Пусть случайная величина $\xi$ имеет нормальное распределение  (см. опред. 12.8)
$$
f_\xi(x)=\frac{1}{\sigma\sqrt{2\pi}}
\exp\left(
	-\frac{(x-a)^2}{2\sigma^2}
\right)
$$

Её характеристическая функция равна
%
%
\begin{gather*}
\Theta\xi (t)=\M e^{it\xi}=
\frac{1}{\sqrt{2\pi}}
\int\limits_{-\infty}^\infty e^{itx}e^{-x^2/2} dx=
%
\frac{1}{\sqrt{2\pi}}
\int\limits_{-\infty}^\infty e^{-t^2/2}e^{-(x-it)^2/2} dx=\\=
e^{-t^2/2}\int\limits_{-\infty}^\infty e^{-(x-it)^2/2} d(x-it)=
e^{-t^2/2}
\end{gather*}

где в показателе экспоненты мы выделили полный квадрат и получили интеграл от функции $\frac{1}{\sqrt{2\pi}}e^{-u^2/2}$.
\end{example}

Напомним:
\begin{repdefinition}{Определение 16.7}
Если две независимые случайные величины имеют одно и то же распределение (возможно, с разными параметрами), и их сумма имеет то же самое распределение, то распределение называется устойчивым относительно суммирования.
\end{repdefinition}

\begin{replemma}{Лемма 16.9}
Биномиальное распределение является устойчивым относительно суммирования.
\end{replemma}
\begin{proof}
\begin{equation*}
	\Theta_{\xi_n +\xi_m}(t) = \theta\xi_n(t)\theta\xi_m(t) = (1-p+pe^{it})^n (1-p+pe^{it})^m=(1-p+pe^{it})^{n+m},
\end{equation*}
следовательно
\begin{equation*}
	\P_(\xi_{n+m} = k) = C_{n+m}^k p^k (1-p)^{n+m-k},
\end{equation*}
где $k = 0, 1, 2, \ldots , n+m$.
\end{proof}


\begin{replemma}{Лемма 16.10}
Нормальное распределение является устойчивым относительно суммирования.
\end{replemma}
\begin{proof}
\begin{gather*}
	\Theta_{\xi_n +\xi_m}(t) = \theta\xi_n(t)\theta\xi_m(t) =
	\exp\left(
		ita_1-\frac{t^2\sigma_1^2}{2}
	\right)
	\exp\left(
		ita_2-\frac{t^2\sigma_2^2}{2}
	\right)
	=\\=
	\exp\left(
		it(a_1+a_2)-\frac{t^2(\sigma_1^2+\sigma_2^2)}{2}
	\right),
\end{gather*}
следовательно
\begin{equation*}
	\P_(\xi_{n+m} = k) = C_{n+m}^k p^k (1-p)^{n+m-k},
\end{equation*}
где $k = 0, 1, 2, \ldots , n+m$.
\end{proof}