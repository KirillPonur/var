%!TEX root = ../var.tex
Одним из главных разделов теории вероятностей составляют так называемые <<предельные теоремы>>. Они описывают условия возникновения вероятностных закономерностей при наличии большого числа случайных факторов. 

Предельные теоремы можно условно разделить на три типа:
1) закон больших чисел, 2) предельные теоремы биномиального распределения и 3) центральные предельные теоремы. 

Этот и следующие параграфы посвящены изучению таких теорем.

\begin{definition}
Говорят, что последовательность случайных величин $\{\xi_n\}^\infty_{n=1}$ удовлетворяет закону больших чисел (ЗБЧ), если для любого $\epsilon \in (0, \infty)$ (при некоторых дополнительных условиях) имеют место эквивалентные друг другу равенства
\end{definition}

\begin{equation}
	\label{lim1}
	\lim\limits_{n\to\infty}
	\P\left(\left|
		\frac{1}{n}\sum\limits_{k=1}^n \xi_k-
		\frac{1}{n}\sum\limits_{k=1}^n \M\xi_k
	\right|<\epsilon\right)=1
\end{equation}
и
\begin{equation}
	\label{lim2}
	\lim\limits_{n\to\infty}
	\P\left(\left|
		\frac{1}{n}\sum\limits_{k=1}^n \xi_k-
		\frac{1}{n}\sum\limits_{k=1}^n \M\xi_k
	\right|\geq\epsilon\right)=0
\end{equation}

Другими словами, среднее арифметическое первых $n$ членов последовательности случайных величин сходится по вероятности к среднему арифметическому математических ожиданий её первых $n$ членов при $n \to \infty$, или в символической форме:
$$
	\frac{1}{n}\sum\limits_{k=1}^n \xi_k
	\stackrel{\P}{\to}
	\frac{1}{n}\sum\limits_{k=1}^n \M\xi_k
	\text{ при $n\to\infty$}
$$

\begin{zam}

Закон больших чисел в обобщённом смысле -- это совокупность теорем о том, что при некоторых условиях имеют место формулы типа (\ref{lim1}) и (\ref{lim2}) и их многочисленные обобщения. 

Т.к. каждая такая теорема представляет собой конкретный закон больших чисел, то имеет смысл говорить о законах больших чисел. 

Широкие условия применимости ЗБЧ были найдены впервые Чебышёвым в 1867 г. Эти условия затем были обобщены А.А. Марковым (старшим). 

Окончательное решение проблемы о 
необходимых и достаточных условиях применимости ЗБЧ было найдено А.Н. Колмогоровым в 1928 г. В этом параграфе мы изучим три наиболее известных таких закона.
\end{zam}

\begin{theorem}[ЗБЧ в форме Чебышёва]
Пусть $\xi_1 , \xi_2 , \ldots , \xi_n , \ldots$ -- последовательность попарно независимых случайных величин, имеющих дисперсии, ограниченные одной и той же константой $C$, т.е.
$$
\D\xi_1 \leq C,\quad
\D\xi_2 \leq C,\quad
\ldots,\quad
\D\xi_n \leq C,\quad
\ldots
$$


Тогда для любого $\epsilon \in (0, \infty)$ при $n \to \infty$
$$
	\frac{1}{n}\sum\limits_{k=1}^n \xi_k
	\stackrel{\P}{\to}
	\frac{1}{n}\sum\limits_{k=1}^n \M\xi_k
$$
\end{theorem}

\begin{proof}
По 18.4.3) и 18.8 имеем	
$$
	\D\left(
		\frac{1}{n}\sum\limits_{k=1}^n \xi_k
	\right)=
	\frac{1}{n^2}\D\sum\limits_{k=1}^n \xi_k=
	\frac{1}{n^2}\sum\limits_{k=1}^n \D\xi_k\leq
	\frac{1}{n^2}Cn=\frac{C}{n}
$$

По след. 22.6
$$
	\P\left(\left|
		\frac{1}{n}\sum\limits_{k=1}^n \xi_k-
		\frac{1}{n}\sum\limits_{k=1}^n \M\xi_k
	\right|<\epsilon\right)
	\geq
	1-\frac{\D\left(\frac{1}{n}\sum\limits_{k=1}^n \xi_k\right)}{\epsilon^2}\geq 1-\frac{C}{n\epsilon^2}
$$

Переходя к пределу при $n \to \infty$ получим неравенство
$$
	\lim\limits_{n\to\infty}
	\P\left(\left|
		\frac{1}{n}\sum\limits_{k=1}^n \xi_k-
		\frac{1}{n}\sum\limits_{k=1}^n \M\xi_k
	\right|<\epsilon\right)\geq 1
$$

откуда по лемме 3.6 получаем требуемый результат.
\end{proof}
\begin{zam}
Из доказательства теоремы 22.3 следует, что двойственное неравенство Бьенеме – Чебышёва можно записать в виде:
\begin{equation}
	\label{bch}
	\P\left(\left|
		\frac{1}{n}\sum\limits_{k=1}^n \xi_k-
		\frac{1}{n}\sum\limits_{k=1}^n \M\xi_k
	\right|\geq\epsilon\right)\leq \frac{\D\left(\frac{1}{n}\sum\limits_{k=1}^n \xi_k\right)}{\epsilon^2}
\end{equation}
\end{zam}

\begin{theorem}[ЗБЧ в форме Хинчина\footnote{Александр Яковлевич Хинчин (1894 -- 1959), один из наиболее значимых математиков в советской школе теории вероятностей. Им получены основополагающие результаты в теории функций действительного переменного, теории чисел, теории вероятностей и статистической физике.} (1929)]

Пусть $\xi_1 , \xi_2 , \ldots , \xi_n , \ldots$ -- последовательность попарно независимых и одинаково распределённых случайных величин, то есть
$$
\M\xi_1 = \M\xi_2 = \ldots = \M\xi_n = \ldots
\quad\text{ и }\quad
\D\xi_1 = \D\xi_2 = \ldots = \D\xi_n = \ldots
$$

Тогда среднее арифметическое $\frac{1}{n}\sum\limits_{k=1}^n\xi_k$ сходится по вероятности к $\M\xi_1$ , другими словами
$$
\frac{1}{n}\sum\limits_{k=1}^n\xi_k \stackrel{\P}{\to} \M\xi_1
$$
При этом для любого $\epsilon \in (0, \infty)$ неравенство Бьенеме – Чебышёва можно записать в виде:
$$
	\P\left(\left|
		\frac{1}{n}\sum\limits_{k=1}^n \xi_k-
		\M\xi_1
	\right|\geq\epsilon\right)\leq \frac{\D\xi_1}{n\epsilon^2}
$$
\end{theorem}

\begin{proof}
Для доказательства достаточно подставить $\frac{1}{n}\sum\limits_{k=1}^n \M\xi_k$ в ЗБЧ в форме Чебышёва и в неравенство (\ref{bch}).
\end{proof}

\begin{theorem}[ЗБЧ в форме Бернулли (1713)]
Пусть $A$ -- событие,
которое может произойти в любом из $n$ независимых испытаний с одной и той же вероятностью $p$. Пусть $n(A)$ -- число появлений события $A$ в этих n испытаниях. Тогда относительная частота $\frac{n(A)}{n}$ сходится по вероятности к вероятности $p$, т.е.
$$
	\frac{n(A)}{n}\stackrel{\P}{\to}\M\xi_1
$$
При этом для любого $\epsilon \in (0, \infty)$ неравенство Бьенеме – Чебышёва можно записать в виде:
$$
	\P\left(\left|
		\frac{n(A)}{n}-
		p
	\right|\geq\epsilon\right)\leq \frac{p(1-p)}{n\epsilon^2}
$$
\end{theorem}
\begin{proof}
Видно, что ЗБЧ в форме Бернулли удовлетворяет условиям ЗБЧ в форме Хинчина. Среднее арифметическое $\frac{1}{n}\sum\limits_{k=1}^n \xi_k$ нулей и единиц, выпавших при $n$ независимых испытаниях, равно частоте появления события $A$ (см. опред 4.2). В примере 18.9.1) мы нашли, что $\M\xi_1 = p$ и $\D\xi_1 = p(1 - p)$. Подставляя эти величины в ЗБЧ в форме Хинчина,
получим ЗБЧ в форме Бернулли.
\end{proof}

\begin{example}
Монету подбрасывают 10 000 раз. Оценить вероятность
того, что относительная частота выпадения герба отличается от классической вероятности 1/2 не менее чем на 0,01.

\textit{Решение}. Другими словами, требуется оценить 
$
	\P\left(\left|
		\frac{n(A)}{n}-
		\frac{1}{2}
	\right|\geq0,01\right)
$
где $n = 10^4$ , $n(A) = \sum\limits_{k=1}^n \xi_k$  -- число выпадений герба при условии, что $\xi_k$ являются независимыми случайными величинами, имеющими распределение Бернулли с $p = \frac{1}{2}$ , и равные единице, если выпал герб, и нулю -- в противном случае. 

Применим ЗБЧ в форме Бернулли. Поскольку $\D\xi_1 = p(1 - p) = \frac{1}{4}$, то искомая оценка сверху выглядит следующим образом:
$$
	\P\left(\left|
		\frac{n(A)}{n}-
		\frac{1}{2}
	\right|\geq0,01\right)\leq
	\frac{\D\xi_1}{n\cdot 0,01^2}=\frac{1}{4\cdot10^4\cdot10^{-4}}=\frac{1}{4}.
$$
Другими словами, неравенство Бьенеме – Чебышёва позволяет заключить, что в среднем не более чем в четверти случаев при 10 000 подбрасываниях монеты частота выпадения герба будет отличаться от 1/2 более чем на 0,01. Мы увидим, что эта оценка достаточно грубая, когда изучим так называемую <<центральную предельную теорему.>>
\end{example}

В заключение параграфа приведём без доказательства ЗБЧ в форме Пуассона.
\begin{theorem}[ЗБЧ в форме Пуассона\footnote{Симеон-Дени Пуассон (Simeon-Denis Poisson, 1781 -- 1840), знаменитый французский физик и математик. Его труды относятся к разным областям чистой математики, математической физики, теоретической
и небесной механики.} (1837)]

Пусть A -- событие, которое может произойти в $n$ независимых испытаниях с вероятностями $p_1 , p_2 , \ldots , p_n$ . Пусть $n(A)$ -- число появлений события A в этих n испытаниях. 

Тогда $\frac{n(A)}{n}$ сходится по вероятности к среднему арифметическому вероятностей $p_k$, т.е.
$$
	\frac{n(A)}{n}\stackrel{\P}{\to}\frac{1}{n}\sum\limits_{k=1}^n p_k
$$

\end{theorem}
