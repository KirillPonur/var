%!TEX root = ../var.tex

В этом параграфе мы подсчитываем число элементарных событий или, проще говоря, исходов, шансов, которые могут возникать в результате эксперимента. 

Например, при подбрасывании монеты могут произойти 2 исхода,
при подбрасывании игрального кубика могут произойти 6 исходов, при извлечении карты из колоды в 54 листа могут произойти 53 исхода. Такие подсчёты изучают в разделе математики, называемом \textit{комбинаторикой}.

Пусть $A$ и $B$ — два непересекающихся конечных множества с числом
элементов $m$ и $n$ соответственно. Очевидны следующие две леммы.



\begin{lemma}(О сумме) 
\label{lemma:1.1}
Число шансов выбрать один элемент либо из $A$
либо из $B$, т.е. из объединения $A\cup B$, равно $m+n$.
\end{lemma}
\begin{lemma}(О произведении) 
\label{lemma:1.2}
Число шансов выбрать пару элементов,
один из $A$, а другой из $B$, равно $mn$, т.е. числу элементов в декартовом
произведении $A\times B$.
\end{lemma}
Непосредственным обобщением предыдущей леммы является следующая
теорема.
\begin{theorem}
\label{th:1.3}
Пусть $A_2,A_2,\dots,A_k$ — конечные непересекающиеся множества, имеющие $n_1
,n_2, \dots, n_k$ элементов соответственно. Выберем из
каждого множества по одному элементу. Тогда общее число способов, которыми можно осуществить такой выбор, равно $n_1n_2\dots n_k$.
\end{theorem}
\begin{proof}

Ясно, что число способов такого выбора равно числу точек (элементов) в декартовом произведении $A_1\times A_2\times\dots A_k$, т.е. равно $
n_1\cdot n_2\dots n_k$.
\end{proof}
\subsection{Эксперименты выбора шариков}

Рассмотрим ящик, содержащий $n$ одинаковых шариков, на которых написаны
числа $1, 2,\dots, n$. Эксперимент состоит в том, что из ящика, не глядя, по
одному вынимают $k$ шариков, где $k\leqslant n$. Обозначим через
\begin{gather*}
(n_1, n_2,\dots, n_k)
\end{gather*}
упорядоченный набор чисел, где $n_1$ — номер 1-го вынутого шарика, $n_2$ —
номер 2-го шарика,$\dots$, $n_k$ — номер $k$-го шарика.

Например, из 5 занумерованных шариков выбрали 3 шарика и получился
набор (4, 2, 1).

Сколько имеется различных способов вынуть из ящика $k$ шариков? На
этот вопрос нельзя дать однозначный ответ, потому что такой эксперимент
определён неоднозначно.

Во-первых, не определено, возвращают ли извлеченный шарик обратно в
ящик. Во-вторых, не определено, какие наборы номеров считать различными
и какие наборы считать одинаковыми.

Рассмотрим следующие возможные условия проведения эксперимента.
\begin{enumerate}
\item 
\textit{Эксперимент с возвращением}. 
Каждый извлечённый шарик возвращается в ящик.
В этом случае в наборе могут появляться одинаковые номера. Например, при выборе трёх шариков из ящика, содержащего пять шариков
с номерами 1, 2, 3, 4 и 5, могут появиться наборы (3, 3, 5), (1, 2, 4)
и (4, 2, 1).
\item 
\textit{Эксперимент без возвращений. Извлечённые шарики в ящик не воз-
вращаются}.
В этом случае в наборе не могут встречаться одинаковые номера. В
рассмотренном выше примере набор (3,3,5) не может появиться, а
наборы (1,2,4) и (4,2,1) могут.

\end{enumerate}

Опишем теперь, какие наборы номеров мы будем считать различными.
Существуют ровно две возможности.
\begin{enumerate}
\item 
\textit{Эксперимент с учётом порядка}. Два набора номеров считаются различными, если они отличаются либо составом, либо порядком.
В рассмотренном выше примере все наборы (3,3,5), (1,2,4) и (4,2,1)
считаются различными.
\item
\textit{Эксперимент без учёта порядка}. Два набора номеров считаются различными, если они отличаются только составом.

\end{enumerate}
В рассмотренном выше примере наборы (1,2,4) и (4,2,1) доставляют
одно и тот же элементарное событие, а набор (3,3,5) — другое.

Подсчитаем теперь, сколько получится различных исходов для каждого
из четырёх экспериментов. Заметим, что в литературе такие эксперименты
часто называют схемами выбора или схемами шансов. Схема шансов -- это
условия (с возвратом или без, какие наборы различны и т.д.), при которых
проводится эксперимент.

\subsection{Схема шансов без возвращения и с учетом порядка}
\begin{theorem}
\label{th:1.4}
В эксперименте без возвращения и с учётом порядка число способов выбрать $k$ элементов из $n$-элементного множества равно
\begin{equation*}
	A_n^k=n(n-1)\dots(n-k+1)=\frac{n!}{(n-k)!}
\end{equation*}
\end{theorem}
Число $A_n^k$ называется \textit{числом размещений элементов $k$ на $n$ местах}. Читается: <<$A$ из $n$ по $k$>>.  

\begin{proof}
При выборе первого шарика имеется $n$ возможностей При выборе первого шарика имеется n возможностей. При выборе второго шарика остаётся $n−1$ возможностей, и т.д. При выборе последнего $k$-го шарика остаётся $n − k + 1$ возможностей. По теор. \ref{th:1.3} общее
число наборов равно $n(n−1)\dots(n−k +1)$, что и требовалось доказать.
\end{proof}

\begin{consq}
\label{cosq:1.5}
Число перестановок из $n$ элементов равно $n!$.
\end{consq}

\begin{proof}
Очевидно, что перестановка есть результат выбора по схеме без возвращения и с учётом порядка всех $n$ элементов из $n$, т.е. общее
число перестановок равно $A_n^n=n!$.
\end{proof}
\subsection{Схема шансов без возвращения и без учёта порядка}
\begin{theorem}
\label{th:1.6}
В эксперименте без возвращения и без учёта порядка число
способов извлечь k из n-элементного множества равно
\begin{equation*}
	C_n^k=\frac{A_n^k}{k!}=\frac{n!}{k!(n-k)!}
\end{equation*}
	
Число $C_n^k$ называется \textit{числом сочетаний k элементов из n элементов.}
Читается: <<$C$ из $n$ по $k$>>
\end{theorem}

\begin{proof}
По следствию \ref{cosq:1.5} из $k$ элементов можно образовать $k!$ упорядоченных наборов. Поэтому количество сочетаний 
(неупорядоченных наборов)
в $k!$ раз меньше, чем число размещений. Поделив $A^k_n$ на $k!$, получим требуемый результат.
\end{proof}

\subsection{Схема шансов с возвращением и с учётом порядка}
\begin{theorem}
\label{th:1.7}
В эксперименте с возвращением и с учётом порядка число
способов извлечь k элементов из n-элементного множества равно $n^k$.
\end{theorem}
\begin{proof}
При выборе каждого из $k$ шариков имеется $n$ возможностей. 
По теореме \ref{th:1.3} общее число наборов равно $n\cdot n\ldots\cdot n=n^k$.
\end{proof}
\subsection{Схема шансов с возвращением и без учёта порядка}

\begin{zam}
\label{zam:1.8}
Рассмотрим для примера ящик с двумя шариками 1 и 2,
из которого мы вынимаем последовательно два шарика. Без учёта порядка
имеется 3 исхода: 
$$\{1,1\}, \{1,2\} = \{2,1\}, \{2, 2\}$$.
\end{zam}

\begin{theorem}
\label{th:1.9}
В эксперименте с возвращением и без учёта порядка число
способов извлечь $k$ элементов из $n$-элементного множества равно $C^k_{n+k-1}$.
\end{theorem}
\begin{proof}
Т.к. порядок появления шариков не учитывается, то мы
учитываем лишь только то, сколько раз в наборе появится $i$-й шарик для
каждого $i = 1, 2,\dots, n$. Обозначим через $k_i$ число появлений $i$-го шарика в
наборе. Во-первых, $0 \leqslant k_i \leqslant k$, а во-вторых,
\begin{gather*}
k_1+k_2+\dots+k_n=k.
\end{gather*}

Поставим каждому исходу в соответствие набор чисел $(k_1, k_2,\dots, k_n)$.
Легко видеть, что это соответствие является взаимно однозначным. Такое соответствие можно рассматривать как способ нумерации наборов. (Например,
исходам из замечания \ref{zam:1.8} ставятся в соответствие следующие номера:
$\{1, 1\} \leftrightarrow (2, 0), 
\{1, 2\} \leftrightarrow (1, 1) \text{ и } 
\{2, 2\} \leftrightarrow (0, 2).$)
Рассмотрим теперь другой эксперимент. Пусть теперь имеется $n$ урн с
номерами $i = 1, 2,\dots, n$, в которых размещаются $k$ неразличимых шариков.
Сколько существует способов разложить шарики по урнам? Нас интересует
только количество шариков в $i$-й урне для каждого $i$. Обозначим через $k_i$
число шариков в $i$-й урне. Ясно, что $0 \leqslant k_i \leqslant k$, и что числа $k_1$ и в этом
эксперименте тоже удовлетворяют уравнению
\begin{gather*}
k_1+k_2+\dots+k_n=k.
\end{gather*}

Исходы этого эксперимента тоже взаимно однозначно описываются наборами
чисел \newline $(k_1, k_2,\dots, k_n)$. Т.о., исходы в эксперименте с урнами и исходы предыдущего эксперимента с ящиком занумерованы одним и тем же набором чисел,
поэтому число исходов в обоих экспериментах одно и то же и равно числу
решений этого уравнения. Вычислим это число для эксперимента с урнами.
Изобразим расположение шариков в урнах с помощью схематичного рисунка. Вертикальными линиями обозначим перегородки между урнами, а
кружками — шарики, находящиеся в них. Например,
\begin{gather*}
\left|\bullet\ \bullet\right|\bullet\bullet\bullet||\bullet||\bullet\bullet|\bullet|.
\end{gather*}

На рисунке показаны 9 шариков, рассыпанные по 7 урнам: 1-я и 6-я урны
содержат по 2 шарика, 2-я урна содержит 3 шарика, 3-я и 5-я урны — пустые
и, наконец, 4-я и 7-я урны содержат по одному шарику.

Меняя местами шарики и стенки, можно получить все возможные расположения шариков в урнах. Другими словами, все расположения можно
получить, расставляя $k$ шариков и $n − 1$ стенок на $n − 1 + k$ местах. Число
$n−1+k$ получается следующим образом. Число стенок у $n$ урн равно $n+1$,
и т.к. две крайние стенки двигать нельзя, то число стенок, которые можно
двигать равно $n − 1$. Поэтому шарики могут занимать $k$ мест, а стенки урн
— оставшиеся $n−1$ место. По теореме \ref{th:1.6} число способов расставить $k$ шариков
на $n − 1 + k$ местах и затем расставить стенки на оставшихся $n − 1$ местах
равно $C^k_{n+k-1}$. Что и требовалось доказать.
\end{proof}
