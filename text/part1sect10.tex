%!TEX root = ../var.tex
\begin{num}
Из урны, содержащей $n_1$ белых и $n - n_1$ чёрных шаров,
вынимают без возвращения $k \leq n$ шаров. Найти вероятность события $A$, состоящего в том, что будет вынуто ровно $k_1$ белых и $k-k_1$ чёрных шаров.
(Мы полагаем, что $k_1 \leq n_1$ и $k - k_1 \leq n - n_1$.)
\end{num}

\textbf{Решение}. Результатом эксперимента является набор из $k$ шаров. Оказывается, что искомая вероятность не зависит от того, будем ли мы или не будем учитывать порядок следования шаров. Чтобы показать это, подсчитаем её для обоих экспериментов.

1. Эксперимент без учёта порядка. В этом эксперименте общее число $\mu(\Omega)$ элементарных исходов равно числу $k$-элементных подмножеств множества, состоящего из $n$ элементов.

По теореме \ref{th:1.6} оно равно $\mu(\Omega)$ = $C_n^k$ .
Обозначим через $A$ событие, состоящее в появлении набора, содержащего$ k_1$ белых шаров и $k - k_1$ чёрных. 

По теореме \ref{th:1.3} число его благоприятных исходов равно произведению числа способов выбрать $k_1$ шаров из $n_1$ белых шаров и числа способов выбрать $k - k_1$ шаров из $n - n_1$ чёрных, т.е $\mu(A) = C_{n_1}^{k_1}\cdot C_{n-n1}^{k-k_1}$.

Получаем, что вероятность появления события $A$ в

\begin{equation}
	P(A)=\frac{C_{n_1}^{k_1}\cdot C_{n-n1}^{k-k_1}}{C_n^k}
\end{equation}

2. Эксперимент с учётом порядка. По теореме \ref{th:1.4} общее число $\mu(\Omega)$ элементарных исходов равно числу способов разместить $n$ элементов на $k$
местах, т.е. $\mu(\Omega) = A_n^k = n(n - 1)\ldots(n - k + 1)$.

В этом эксперименте при подсчёте числа благоприятных исходов $\mu(A)$ надо учесть как число способов выбрать нужное число $k$ шаров, так и число способов расположить белые и чёрные шары среди выбранных $k$ шаров.

Во-первых, мы подсчитываем число способов расположить $k_1$шаров на $k$
местах. Оно равно $C_k^{k_1}$. 

Затем мы подсчитываем число способов расположить $k_1$ белых шаров на $n_1$ местах. Учитывая их порядок, получаем, что
это число равно $A_{n_1}^{k_1}$. 

И наконец, мы подсчитываем число способов разместить $k - k_1$ чёрных шаров на оставшихся $n - n_1$ местах. Оно равно $A_{n-n_1}^{k-k_1}$.

Перемножая эти числа, по теореме \ref{th:1.4} получим
\begin{equation}
	\mu(A) = C_{k}^{k_1}\cdot A_{n_1}^{k_1}\cdot A_{n-n1}^{k-k_1}
\end{equation}

Подсчитывая искомую вероятность, получим
\begin{equation}
	P(A)=\frac{C_{k}^{k_1}\cdot A_{n_1}^{k_1}\cdot A_{n-n1}^{k-k_1}}{A_n^k}=\frac{C_{n_1}^{k_1}\cdot C_{n-n1}^{k-k_1}}{C_n^k}
\end{equation}

\begin{definition}
Ряд распределения, определённый соответствием
\begin{equation}
	k_1 \mapsto P(A) = \frac{C_{n_1}^{k_1}\cdot C_{n-n1}^{k-k_1}}{C_n^k},
\end{equation}
где $0 \leq k_1 \leq \min(k, n_1 )$ и $k - k_1 \leq n - n_1$ называется \textit{гипергеометрическим распределением}.
\end{definition}
