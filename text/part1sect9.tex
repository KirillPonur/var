%!TEX root = ../var.tex

Полиномиальное распределение возникает в результате повторения $n$ раз
схемы Бернулли (см. \S\ref{par:8}). По аналогии $k$-номиальное распределение возникает в результате повторения $n$ раз схемы независимых испытаний с $k \geqslant 3$
исходами (см. пп. \ref{def:4.5} и \ref{ex:4.9}).

\textbf{Пример}. Асимметричный тетраэдр, грани которого обозначены $\omega_1$---$\omega_4$ подбрасывают 14 раз. Вероятности выпадения этих граней лицом вниз
соответственно равны $\frac{1}{2},\frac{1}{3},\frac{1}{9},\frac{1}{18}$. Найти вероятность того, что грани $\omega_1, \omega_2, \omega_3, \omega_4$ выпадут соответственно 5, 3, 4, 2 раза.

\textit{Ответ: }$\frac{14!}{5!3!4!2!}(\frac{1}{2})^5,(\frac{1}{3})^3,(\frac{1}{9})^4,(\frac{1}{18})^2\simeq0,00137$

\begin{zam}
\label{zam:9.1}
	Известно, что биномом Ньютона называют формулу
	\begin{equation*}
		(p+q)^n	=\sum\limits^n_{m=0}C_n^mp^mq^{n-m},
	\end{equation*}
	где $C_n^m=\frac{n!}{m!(n-m)!}$


Обобщением бинома Ньютона является следующая формула
\begin{equation*}
	(p_1+p_2+\ldots+ p_k)^n=\sum\limits_{m_1+m_2+\ldots+m_k}\frac{n!}{m_1!m_2!\ldots m_k!}p_1^{m_1}p_2^{m_2}\ldots p_k^{m_k},
\end{equation*}
где $m_1, m_2,\ldots, m_k \geqslant 0$. Эта формула может быть выведена из бинома Нью-
тона индукцией по k; по аналогии будем назвать её \textit{k-номом Ньютона}.

\end{zam}

\begin{lemma}
\label{lemma:9.2}

	Сумма $k$-номиальных коэффициентов равна $k^n$, т.е.
	\begin{equation*}
		\sum\limits_{m_1+m_2+\ldots+m_k}\frac{n!}{m_1!m_2!\ldots m_k!}=k^n
	\end{equation*}

\end{lemma}

\begin{proof}
Подставим в $k$-ном Ньютона $p_1 = p_2 = \ldots = p_k = 1$,
получим требуемый результат.
\end{proof}

\begin{lemma}
\label{lemma:9.3}

	Если $p_1+p_2+\ldots+p_k = 1$, то сумма членов в правой части
k-нома Ньютона равна единице, т.е.
\begin{equation*}
	\sum\limits_{m_1+m_2+\ldots+m_k=n}\frac{n!}{m_1!m_2!\ldots m_k!}p_1^{m_1}p_2^{m_2}\ldots p_k^{m_k}=1
\end{equation*}
\end{lemma}

\begin{proof}
	Подставим в $k$-ном Ньютона $p_1+p_2+\ldots+p_k = 1$,
получим требуемый результат.
\end{proof}

\begin{repdefinition}{Определение 9.3}


	Пусть эксперимент состоит в том, что

1) проводят n независимых испытаний в одинаковых условиях,

2) в каждом испытании появляется одно из k несовместных событий
{$$A_1, A_2,\ldots, A_k,$$}

3) эти события происходят с вероятностями $p_1, p_2,\ldots, p_k$ соответственно,

и 4) $p_1 + p_2 + \ldots+ p_k = 1$.

Тогда такой эксперимент называется схемой независимых испытаний.
\end{repdefinition}

\begin{theorem}
\label{th:9.5}

	Для любой схемы n независимых испытаний и любых \newline
$m_1 \geqslant 0, m_2 \geqslant 0,\ldots  , m_k \geqslant 0$, таких что $m_1 + m_2 +\ldots  + m_k = n$, вероятность $\P(m_1,m_2,\ldots  ,m_k)$ того, что события $A_1, A_2,\ldots  , A_k$ произойдут
соответственно $m_1,m_2,\ldots  ,m_k$ раз, определяется по формуле
	\begin{equation*}
		\P(m_1,m_2,\ldots,m_k)=\frac{n!}{m_1!m_2!\ldots m_k!}p_1^{m_1}p_2^{m_2}\ldots p_k^{m_k}.
	\end{equation*}
\end{theorem}

\begin{proof}
Рассмотрим один элементарный исход схемы $n$ независимых испытаний:
\begin{equation*}
	(\underbrace{A_1,\ldots , A_1}_{m_1 \text{ раз}}, 
	\underbrace{A_2, \ldots,A_2}_{m_2 \text{ раз}},
	\ldots
	\underbrace{A_k,\ldots , A_k}_{m_k \text{ раз}})
\end{equation*}
Это один из благоприятных исходов: сначала событие $A_1$ произошло $m_1$ раз,
затем событие $A_2$ произошло $m_2$ раз, \ldots, и, наконец, событие $A_k$ произошло
$m_k$ раз. Вероятность этого элементарного исхода равна 
$p^{m_1}_1p^{m_2}_2\ldots p^{m_k}_k$.

Все остальные благоприятные исходы отличаются лишь расположением
событий из того же набора событий на $n$ местах. Число таких исходов равно
числу способов расставить на $n$ местах $m_1$ событий $A_1$, потом $m_2$ событий
$A_2, \ldots$, и, наконец, $m_k$ событий $A_k$. По теор. \ref{th:1.3} и \ref{th:1.6} это число равно
\begin{equation*}
	C_n^{m_1}\cdot C_{n-m_2}^{m_2}\cdot C_{n-m_2-m_2}^{m_3}\cdot\ldots\cdot
	C_{n-m_2-\ldots m_{k-1}}^{m_k}=
	\left[\text{проверьте это!} \right]=\frac{n!}{m_1!m_2!\ldots m_k!}
\end{equation*}
\end{proof}
\textit{Теперь становится ясно, откуда взялся ответ задачи про асимметрич-
ный тетраэдр в начале этого параграфа.}