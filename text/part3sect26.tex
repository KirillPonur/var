%!TEX root = ../var.tex


\begin{zam}
Центральные предельные теоремы (ЦПТ) в теории вероятностей -- это класс теорем, утверждающих, что сумма большого количества независимых случайных величин имеет распределение близкое к нормальному. 

Так как многие случайные явления в природе и социальном обществе являются суммами большого числа случайных факторов, то центральные предельные теоремы обосновывают фундаментальность нормального закона распределения. История получения ЦПТ растянулась на два века -- от первых работ Муавра 1730 г. до необходимых и достаточных условий, полученных в 30-х гг. XX века. 

Самый общий классический случай ЦПТ для неодинаково распределённых последовательностей случайных величин принадлежит А.М. Ляпунову\footnote{Александр Михайлович Ляпунов (1857 -- 1918), выдающийся русский математик, основные труды
по устойчивости динамических систем, теории потенциала и механике. Занятие теорией вероятностей было кратким эпизодом в его научной работе, тем не менее в этой области он добился фундаментальных результатов. Он доказал ЦПТ при весьма широких условиях методом, который и сейчас является одним
из основных в теории вероятностей.} (1901). Мы докажем ЦПТ в классической форме (теор. 26.3), а чтобы иметь представление о ЦПТ в форме Ляпунова (теор. 26.5) мы приведём её без доказательства. 
\end{zam}

\begin{zam}
Вернёмся ещё раз к ЗБЧ в форме Хинчина. Пусть $$\xi_1 , \xi_2 , \ldots , \xi_n , \ldots$$ -- последовательность попарно независимых одинаково распределённых случайных величин, т.е. все имеющие математическое ожидание $\M\xi_1$ и дисперсию $\D\xi_1$. Обозначим $S_n=\sum\limits_{k=1}^n\xi_k$. 

Тогда по ЗБЧ в форме Хинчина среднее арифметическое $\frac{S_n}{n}$ сходится по вероятности к $\M\xi_1$, или, что то же самое, величина $\frac{S_n-n\M\xi_1}{n}$ сходится по вероятности к нулю. 

Запишем это утверждение с помощью неравенства Бьенеме – Чебышёва:
$$
\P\left( \left|\frac{S_n-n\M\xi_1}{n}\right| < \epsilon \right) \geq 1-\frac{\D\xi_1}{\epsilon^2}
$$

Т.к. вероятность не превосходит 1, то это неравенство можно переписать в виде
$$
1-\frac{\D\xi_1}{\epsilon^2} \geq \P\left( \left|\frac{S_n-n\M\xi_1}{n}\right| < \epsilon \right) \leq 1 
$$

Если теперь перейти к пределу при $n \to \infty$, то левая часть этого сквозного неравенства стремится к единице, и вероятность, стоящая в средней части, <<намертво>> впечатывается в единицу. В этом и состоит высший
смысл Закона Больших Чисел: при стремлении $n$ к бесконечности и для любого $\epsilon > 0$ неравенство  $\left|\frac{S_n-n\M\xi_1}{n}\right|  < \epsilon$ становится достоверным событием. 

Поэтому ЗБЧ есть строгое математическое выражение свойства
статистической устойчивости (см. опред 4.4). Заметим, что $\M\xi_1$ и $\D\xi_1$ являются константами, а $\epsilon > 0$ хотя и любое, но тоже постоянное число.

Идея как-то оторвать упомянутую вероятность от единицы заключается в следующем: а не слишком ли на большую степень числа n мы поделили в выражении $\frac{S_n-n\M\xi_1}{n}$?

Нельзя ли поделить на что-нибудь, растущее к бесконечности
медленнее чем $n$ так, чтобы последовательность событий
$\left\{\left|\frac{S_n-n\M\xi_1}{n^\alpha}\right|  < \epsilon\right\}$
при некотором $0 < \alpha < 1$ перестала бы сходиться к достоверному (и, само собой разумеется, не сходилась бы к невозможному событию)? 

Оказывается, если выбрать $\alpha = 1/2$, то при $n \to \infty$ последовательность случайных величин $\eta_n = \frac{S_n -n\M\xi}{\sqrt{n}}$ слабо сходится к случайной величине $\eta$, имеющей нормальное (!) распределение.
\end{zam}

\begin{theorem}[ЦПТ в классической форме]
Если случайные величины $\xi_1 , \xi_2 , \ldots , \xi_n , \ldots$ независимы, одинаково распределены и имеют конечные математическое ожидание $\M_{\xi} = a < \infty$ и дисперсию $\D_{\xi} = \sigma^2 < \infty$ и $\sigma \ne 0$, тогда последовательность случайных величин
$$
\eta_n =
\frac{
	\xi_1 + \ldots + \xi_n - na
}{\sqrt{\sigma n}}
$$

слабо сходится к случайной величине, имеющей стандартное нормальное распределение, т.е.
$$
\lim\limits_{n\to\infty}
\P\left(
	\frac{
		\xi_1 + \ldots + \xi_n - na
	}{\sqrt{\sigma n}}\leq x
\right)
=\frac{1}{\sqrt{2\pi}}\int\limits_{-\infty}^x e^{-u^2/2} du,
$$
или кратко
$$
\lim\limits_{n\to\infty}
\P\left(
	\eta_n \leq x
\right)
=\frac{1}{\sqrt{2\pi}}\int\limits_{-\infty}^x e^{-u^2/2} du,
$$
\end{theorem}
\begin{proof}

Доказательство. Для упрощения выкладок вместо $\xi_1 , \xi_2 , \ldots , \xi_n , \ldots$
введём стандартные случайные величины $\oxi_1 , \oxi_2 , \ldots , \oxi_n , \ldots$ по формулам
$$
\oxi_i=\frac{\xi_i-a}{\sigma}
$$

Легко видеть, что случайные величины $\oxi_i$ являются независимыми, одинаково распределёнными, имеют нулевое математическое ожидание и единичную дисперсию и следовательно имеют одинаковые характеристические функции. Обозначим их характеристические функции тем же символом как и для первой случайной величины $\theta_{\oxi_i}(t) = \theta_{\xi_1} (t)$. Используя теор. 24.12, разложим её в ряд Тейлора, содержащий три первых члена плюс остаточный член:
$$
\theta_{\oxi_i}(t)=1-\frac{1}{2}t^2+o(t^2),
$$

где мы учли, что $\M\oxi_1 = 0$ и $\M\oxi_1^2 = \D\oxi_1 + (\M\oxi_1)^2 = 1$.

Случайная величина $\eta_n$ может быть записана в виде
$$\eta_n =\frac{\oxi_1+\ldots+\oxi_n}{\sqrt{n}}=
\frac{\oxi_1}{\sqrt{n}}+\ldots+\frac{\oxi_n}{\sqrt{n}}
$$

Запишем и преобразуем её характеристическую функцию, получим
\begin{gather*}
\theta\eta_n (t) = \theta_{\frac{\oxi_1}{\sqrt{n}} + \ldots + \frac{\oxi_1}{\sqrt{n}}}(t)
\stackrel{24.8}{=}
\theta_{\frac{\oxi_1}{\sqrt{n}}}(t)\cdot\ldots\cdot \theta_{\frac{\oxi_1}{\sqrt{n}}}(t)
\stackrel{24.6}{=}\\=
\theta_{\oxi_1}\left(\frac{t}{\sqrt{n}}\right)
	\cdot\ldots\cdot
\theta_{\oxi_n}\left(\frac{t}{\sqrt{n}}\right)=
\left[
	\theta_{\oxi_1}\left(\frac{t}{\sqrt{n}}\right)
\right]^n=
\left[
	1-\frac{t^2}{2n}+o\left(\frac{t^2}{n}\right)
\right]^n
\end{gather*}

Т.к. по второму замечательному пределу имеем
$$
\lim\limits_{n\to\infty}
\left[
	1-\frac{t^2}{2n}+o\left(\frac{t^2}{n}\right)
\right]^n=e^{-t^2/2}
$$
то это означает, что последовательность характеристических функций
${\{\theta\eta_n (t)\}}_{n=1}^\infty$ поточечно сходится к характеристической функции 
${\theta\eta_n (t)}=e^{-t^2/2}$.

По пункту 25.5 эта функция является характеристической функцией стандартного нормального распределения с плотностью вероятности
$f_\eta (x) = \frac{1}{\sqrt{2\pi}}e^{-x^2/2}$. 

По теор. 24.13 (Леви о непрерывном соответствии), последовательность случайных величин ${\{\eta_n \}}^\infty_{n=1}$ слабо сходится к случайной
величине $\eta$, а по опред. 21.9 слабой сходимости имеем
\begin{gather*}
\lim\limits_{n\to\infty} \P(\eta_n \leq x)
\stackrel{11.3}{=}
\lim\limits_{n\to\infty} F_{\eta_n}(x)
\stackrel{21.9}{=}
F_\eta(x)
\stackrel{12.1}{=}
\frac{1}{\sqrt{2\pi}}\int\limits_{-\infty}^x e^{-u^2/2} du.
\end{gather*}
Что и требовалось доказать.
\end{proof}

\begin{consq}
Если случайные величины $\xi_1 , \xi_2 , \ldots , \xi_n , \ldots$ независимы, одинаково распределены и имеют конечную ненулевую дисперсию и
$\eta_n =\frac{\xi_1+\xi_2+\ldots+\xi_n-na}{\sigma\sqrt{\pi}},$
то следующие утверждения эквивалентны ЦПТ в классической форме.

1) Для любых $x < y$ имеет место равенство:
$$
\lim\limits_{n\to\infty} \P(x<\eta_n<y)=
\frac{1}{\sqrt{2\pi}}\int\limits_{x}^y e^{-u^2/2} du.
$$

2) Для любых $x < y$ имеет место равенство:
$$
\lim\limits_{n\to\infty} \P(x \leq \eta_n \leq y)=
\frac{1}{\sqrt{2\pi}}\int\limits_x^y e^{-u^2/2} du.
$$

\end{consq}

\begin{theorem}[ЦПТ в форме Ляпунова]
Пусть случайные величины $$\xi_1 , \xi_2 , \ldots , \xi_n , \ldots$$ независимы и имеют конечные абсолютные начальные моменты 3-го порядка $(\M|\xi_n|^3 < \infty)$. Введём следующие обозначения:
$$
A_n=\sum\limits_{k=1}^n \M\xi_k,
\quad
B_n^2=\sum\limits_{k=1}^n \D\xi_k,
\quad
C_n^3=\sum\limits_{k=1}^n \M|\xi_k-\M\xi_k|^3.
$$

Если
$$
\lim\limits_{n\to\infty}\frac{C_n}{B_n}=0,
$$
тогда последовательность случайных величин
$$
\frac{\xi_1+\xi_2+\ldots+\xi_n-A_n}{B_n}
$$
слабо сходится к случайной величине, имеющей стандартное нормальное
распределение, т.е.
$$
\lim\limits_{n\to\infty} \P\left(\frac{\xi_1+\xi_2+\ldots+\xi_n-A_n}{B_n} \leq x\right)=
\frac{1}{\sqrt{2\pi}}\int\limits_{-\infty}^x e^{-u^2/2} du.
$$
\end{theorem}

В качестве следствия из ЦПТ докажем предельную теорему Муавра-Лапласа. Подобно ЗБЧ в форме Бернулли предельная теорема Муавра-Лапласа является утверждением только для схемы Бернулли. Напомним её формулировку.

\begin{theorem}[Интегральная теорема Муавра-Лапласа]
Если случайная величина $\eta_n$ имеет биномиальное распределение $\P(\eta_n = k) = C_n^k p^k q^{n-k}$, тогда для любых вещественных $k_1$ и $k_2$ имеет место равенство
$$
\lim\limits_{n\to\infty} \P\left(k_1\leq\eta_n\leq k_2\right)=
\frac{1}{\sqrt{2\pi}}\int\limits_{x}^{y} e^{-u^2/2} du,
$$
где $x=\frac{k_1-np}{\sqrt{npq}}$ и $y=\frac{k_2-np}{\sqrt{npq}}$.
\end{theorem}
\begin{proof}
Приведём обозначения теоремы Муавра-Лапласа в соответствие с обозначениями ЦПТ. Случайная величина $\xi$ есть число появлений события $A$ в результате $n$ испытаний в схеме Бернулли с вероятностью
$\P(A) = p$. 

Если обозначить через $\xi_i$ число (равное 0 или 1) появлений события $A$ в результате $i$-го испытания, где$ i = 1, \ldots , n,$ то случайную величину $\eta_n$ можно представить в виде суммы $\eta_n = \xi_1 + \ldots + \xi_n$ независимых, одинаково распределённых случайных величин, имеющих одинаковые конечные математические ожидания $\M_{\xi_i} = $p и дисперсии $\D_{\xi_i} = p(1 - p) = pq$. 

Тогда по пункту 18.9.2) имеем $\M\eta_n = np$ и $\D\eta_n = npq$. 

Заметим, что неравенство $k_1\leq\eta_n\leq k_2$
эквивалентно неравенству
$$
\frac{k_1-np}{\sqrt{npq}}\leq
\frac{\eta_n-np}{\sqrt{npq}}\leq
\frac{k_2-np}{\sqrt{npq}}.
$$
По следствию 26.4.2) получим требуемый результат.
\end{proof}

\begin{example}[Ср. пример 21.7]
Монету подбрасывают 10 000 раз. Найти вероятность того, что относительная частота выпадения герба отличается от классической вероятности $1/2$ не менее чем на $0,01$.


\textbf{Решение}. Другими словами, требуется найти $\P\left(\left|\frac{n(A)}{n}-\frac{1}{2}\right|\geq0,01\right)$, где $n = 10^4$ , $n(A) =
\sum\limits_{k=1}^n \xi_k$ -- число выпадений герба, а $\xi_k$ являются независимыми случайными величинами, имеющие одно и то же распределение
Бернулли с $p = \frac{1}{2}$, и равные единице, если выпал герб, и нулю -- в противном случае. 

Умножим обе части неравенства под знаком вероятности на $\sqrt{n} = 100$ и разделим на корень из дисперсии $\sqrt{\D\xi_1} = \sqrt{pq} = \frac{1}{2}$.
\begin{gather*}
	\P\left(
		\left|
			\frac{n(A)}{n}-p
		\right|\geq0,01
	\right)=
	1-\P\left(
		\left|
			\frac{n(A)-np}{n}
		\right|<0,01
	\right)=\\=
	1-\P\left(
		\frac{\sqrt{n}}{\sqrt{\D\xi_1}}
		\left|
			\frac{n(A)-np}{n}
		\right|<0,01\frac{\sqrt{n}}{\sqrt{\D\xi_1}}
	\right)=\\=
	1-\P\left(
		\left|
			\frac{n(A)-np}{\sqrt{npq}}
		\right|<2
	\right)=
	\frac{1}{\sqrt{2\pi}}\int\limits_{-2}^2 e^{-u^2/2} du=\\=
	1-2\Phi(2)\simeq 1 - 2 \cdot 0,47725 = 0,0455,
\end{gather*}
где значение $\Phi(2)$ интеграла вероятности взято из таблицы. Заметим,
что применение ЦПТ и предельной теоремы Муавра-Лапласа доставляет
лучшую оценку чем ЗБЧ в форме Бернулли (см. ответ в примере 21.7.).

\end{example}
