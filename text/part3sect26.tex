%!TEX root = ../var.tex

§26. Центральная предельная теорема
\begin{zam}
Центральные предельные теоремы (ЦПТ) в теории вероятностей -- это класс теорем, утверждающих, что сумма большого количества независимых случайных величин имеет распределение близкое к нормальному. 

Так как многие случайные явления в природе и социальном обществе являются суммами большого числа случайных факторов, то центральные предельные теоремы обосновывают фундаментальность нормального закона распределения. История получения ЦПТ растянулась на два века -- от первых работ Муавра 1730 г. до необходимых и достаточных условий, полученных в 30-х гг. XX века. 

Самый общий классический случай ЦПТ для неодинаково распределённых последовательностей случайных величин принадлежит А.М. Ляпунову\footnote{Александр Михайлович Ляпунов (1857 -- 1918), выдающийся русский математик, основные труды
по устойчивости динамических систем, теории потенциала и механике. Занятие теорией вероятностей было кратким эпизодом в его научной работе, тем не менее в этой области он добился фундаментальных результатов. Он доказал ЦПТ при весьма широких условиях методом, который и сейчас является одним
из основных в теории вероятностей.} (1901). Мы докажем ЦПТ в классической форме (теор. 26.3), а чтобы иметь представление о ЦПТ в форме Ляпунова (теор. 26.5) мы приведём её без доказательства. 
\end{zam}

\begin{zam}
Вернёмся ещё раз к ЗБЧ в форме Хинчина. Пусть $\xi_1 , \xi_2 , \ldots , \xi_n , \ldots$ -- последовательность попарно независимых одинаково распределённых случайных величин, т.е. все имеющие математическое ожидание $\M\xi_1$ и дисперсию $\D\xi_1$. Обозначим $S_n=\sum\limits_{k=1}^n\xi_k$. 

Тогда по ЗБЧ в форме Хинчина среднее арифметическое $\frac{S_n}{n}$ сходится по вероятности к $\M\xi_1$, или, что то же самое, величина $\frac{S_n-n\M\xi_1}{n}$ сходится по вероятности к нулю. 

Запишем это утверждение с помощью неравенства Бьенеме – Чебышёва:
$$
\P\left( \left|\frac{S_n-n\M\xi_1}{n}\right| < \epsilon \right) \geq 1-\frac{\D\xi_1}{\epsilon^2}
$$

Т.к. вероятность не превосходит 1, то это неравенство можно переписать в виде
$$
1-\frac{\D\xi_1}{\epsilon^2} \geq \P\left( \left|\frac{S_n-n\M\xi_1}{n}\right| < \epsilon \right) \leq 1 
$$

Если теперь перейти к пределу при $n \to \infty$, то левая часть этого сквозного неравенства стремится к единице, и вероятность, стоящая в средней части, <<намертво>> впечатывается в единицу. В этом и состоит высший
смысл Закона Больших Чисел: при стремлении $n$ к бесконечности и для любого $\epsilon > 0$ неравенство  $\left|\frac{S_n-n\M\xi_1}{n}\right|  < \epsilon$ становится достоверным событием. 

Поэтому ЗБЧ есть строгое математическое выражение свойства
статистической устойчивости (см. опред 4.4). Заметим, что $\M\xi_1$ и $\D\xi_1$ являются константами, а $\epsilon > 0$ хотя и любое, но тоже постоянное число.

Идея как-то оторвать упомянутую вероятность от единицы заключается в следующем: а не слишком ли на большую степень числа n мы поделили в выражении $\frac{S_n-n\M\xi_1}{n}$?

Нельзя ли поделить на что-нибудь, растущее к бесконечности
медленнее чем $n$ так, чтобы последовательность событий
$\left\{\left|\frac{S_n-n\M\xi_1}{n^\alpha}\right|  < \epsilon\right\}$
при некотором $0 < \alpha < 1$ перестала бы сходиться к достоверному (и, само собой разумеется, не сходилась бы к невозможному событию)? 

Оказывается, если выбрать $\alpha = 1/2$, то при $n \to \infty$ последовательность случайных величин $\eta_n = \frac{S_n -n\M\xi}{\sqrt{n}}$ слабо сходится к случайной величине $\eta$, имеющей нормальное (!) распределение.
\end{zam}

\begin{theorem}[ЦПТ в классической форме]
Если случайные величины $\xi_1 , \xi_2 , \ldots , \xi_n , \ldots$ независимы, одинаково распределены и имеют конечные математическое ожидание $\M_{\xi} = a < \infty$ и дисперсию $\D_{\xi} = \sigma^2 < \infty$ и $\sigma \ne 0$, тогда последовательность случайных величин
$$
\eta_n =
\frac{
	\xi_1 + \ldots + \xi_n - na
}{\sqrt{\sigma n}}
$$

слабо сходится к случайной величине, имеющей стандартное нормальное распределение, т.е.
$$
\lim\limits_{n\to\infty}
\P\left(
	\frac{
		\xi_1 + \ldots + \xi_n - na
	}{\sqrt{\sigma n}}\leq x
\right)
=\frac{1}{\sqrt{2\pi}}\int\limits_{-\infty}^x e^{-u^2/2} du,
$$
или кратко
$$
\lim\limits_{n\to\infty}
\P\left(
	\eta_n \leq x
\right)
=\frac{1}{\sqrt{2\pi}}\int\limits_{-\infty}^x e^{-u^2/2} du,
$$
\end{theorem}
\begin{proof}

Доказательство. Для упрощения выкладок вместо $\xi_1 , \xi_2 , \ldots , \xi_n , \ldots$
введём стандартные случайные величины $\oxi_1 , \oxi_2 , \ldots , \oxi_n , \ldots$ по формулам
$$
\oxi_i=\frac{\xi_i-a}{\sigma}
$$

\sigma

Легко видеть, что случайные величины $\oxi_i$ являются независимыми, одинаково распределёнными, имеют нулевое математическое ожидание и единичную дисперсию и следовательно имеют одинаковые характеристические функции. Обозначим их характеристические функции тем же символом как и для первой случайной величины $\Theta_{\oxi_i}(t) = \Theta_{\xi_1} (t)$. Используя теор. 24.12, разложим её в ряд Тейлора, содержащий три первых члена плюс остаточный член:
$$
\Theta_{\oxi_i}(t)=1-\frac{1}{2}t^2+o(t^2),
$$

где мы учли, что $\M\oxi_1 = 0$ и $\M\oxi_1^2 = \D\oxi_1 + (\M\oxi_1)^2 = 1$.

Случайная величина $\eta_n$ может быть записана в виде
$$\eta_n =\frac{\oxi_1+\ldots+\oxi_n}{\sqrt{n}}=
$$

Запишем и преобразуем её характеристическую функцию, получим
24.8

24.6

\Theta\eta_n (t) = \Theta \sqrt\xi_1 + \ldots + \sqrt\xi_n (t) = \Theta \sqrt\xi_1 (t) \cdot \ldots \cdot \Theta \sqrt\xi_n (t) =
n

(
= \Theta \xi_1

t
\sqrt
n

)

n

(
\cdot \ldots \cdot \Theta \xi_n

t
\sqrt
n

n

n

)

[
= \Theta \xi_1

(

t
\sqrt
n

)]n

[
( 2 )]n
t2
t
= 1-
+o
.
2n
n

Т.к. по второму замечательному пределу имеем
[
( 2 )]n
t2
t
2
\lim 1 -
+o
= e-t /2 ,
n\to\infty
2n
n
то это означает, что последовательность характеристических функций
-t2 /2
\{\Theta\eta_n (t)\}\infty
.
n=1 поточечно сходится к характеристической функции \Theta\eta (t) = e
По пункту 25.5 эта функция является характеристической функцией стандартного нормального распределения с плотностью вероятности
2
f\eta (x) = \sqrt12\pi e-x /2 . По теор. 24.13 (Леви о непрерывном соответствии), последовательность случайных величин \{\eta_n \}\infty
n=1 слабо сходится к случайной
величине \eta, а по опред. 21.9 слабой сходимости имеем
1
11.3
21.9
12.1
\lim \P ( \eta_n \leq x ) = \lim F_\eta_n (x) = F_\eta (x) = \sqrt
n\to\infty
n\to\infty
2\pi

\intx

e-u

2

/2

du.

-\infty

Что и требовалось доказать.
\end{proof}

\begin{consq}
Если случайные величины \xi_1 , \xi_2 , \ldots , \xi_n , \ldots независимы, одинаково распределены и имеют конечную ненулевую дисперсию и
n -na
\eta_n = \xi_1 + . .\sigma.\sqrt+\xi
, то следующие утверждения эквивалентны ЦПТ в класn
сической форме.
1) Для любых x < y имеет место равенство:
1
\lim \P ( x < \eta_n < y ) = \sqrt
n\to\infty
2\pi

\inty

e-u

2

/2

du.

/2

du.

x

2) Для любых x < y имеет место равенство:
1
\lim \P ( x \leq \eta_n \leq y ) = \sqrt
n\to\infty
2\pi
89

\inty
x

e-u

2	
\end{consq}

\begin{theorem}[ЦПТ в форме Ляпунова]
Пусть случайные величины
\xi_1 , \xi_2 , \ldots , \xi_n , \ldots независимы и имеют конечные абсолютные начальные
моменты 3-го порядка (\M|\xi_n |3 < \infty). Введём следующие обозначения
A_n =

n
\sum

\M\xi_k ,

k=1

Bn2

=

n
\sum

C_n3

\D\xi_k ,

=

k=1

n
\sum

\M|\xi_k - \M\xi_k |3 .

k=1

Если

C_n
= 0,
n\to\infty Bn
тогда последовательность случайных величин
\lim

\xi_1 + \ldots + \xi_n - A_n
Bn
слабо сходится к случайной величине, имеющей стандартное нормальное
распределение, т.е.
(

\xi_1 + \ldots + \xi_n - A_n
\leqx
Bn

\lim \P

n\to\infty

)

1
=\sqrt
2\pi

\intx

e-u

2

/2

du.

-\infty
\end{theorem}


Доказательство теоремы Муавра-Лапласа 26.8
В качестве следствия из ЦПТ докажем предельную теорему МуавраЛапласа. Подобно ЗБЧ в форме Бернулли предельная теорема Муавра-Лапласа
является утверждением только для схемы Бернулли. Напомним её формулировку.
Теорема 26.8 (интегральная теорема Муавра-Лапласа). Если случайная величина \eta_n имеет биномиальное распределение \P(\eta_n = k) = C_n^k pk q n-k ,
тогда для любых вещественных k1 и k2 имеет место равенство
\inty
\lim \P(k1 \leq \eta_n \leq k2 ) =

n\to\infty

x

где x =

k1 -np
\sqrt
npq

иy=

1
2
\sqrt e-t /2 dt,
2\pi

k2 -np
\sqrt
npq .

Доказательство. Приведём обозначения теоремы Муавра-Лапласа в соответствие с обозначениями ЦПТ. Случайная величина \xi есть число появлений события A в результате n испытаний в схеме Бернулли с вероятностью
\P(A) = p. Если обозначить через \x_{ii} число (равное 0 или 1) появлений события A в результате i-го испытания, где i = 1, \ldots , n, то случайную величину
90

\eta_n можно представить в виде суммы \eta_n = \xi_1 + \ldots + \xi_n независимых, одинаково распределённых случайных величин, имеющих одинаковые конечные
математические ожидания \M_\x_{ii} = p и дисперсии \D_\x_{ii} = p(1 - p) = pq. Тогда
по пункту 18.9.2) имеем \M\eta_n = np и \D\eta_n = npq. Заметим, что неравенство
\eta_n -np
k2 -np
\sqrt
\sqrt
k1 \leq \eta_n \leq k2 эквивалентно неравенству k\sqrt1 -np
npq \leq
npq \leq
npq . По след.
26.4.2) получим требуемый результат.
Пример 26.6. (Ср. пример 26.7.) Монету подбрасывают 10 000 раз.
Найти вероятность того, что относительная частота выпадения герба
отличается от классической вероятности 1/2 не менее
(  чем на 0,01.
)
 n(A) 1 
Решение. Другими словами, требуется найти \P  n - 2  \geq 0, 01 ,
n
\sum
где n = 104 , n(A) =
\xi_k -- число выпадений герба, а \xi_k являются незаk=1

висимыми случайными величинами, имеющие одно и то же распределение
Бернулли с p = 12 , и равные единице, если выпал герб, и нулю -- в противном
вероятности на
\sqrt случае. Умножим обе части неравенства\sqrtпод знаком
\sqrt
n = 100 и разделим на корень из дисперсии \D\xi_1 = pq = 21 .


(
)
(
)

 n(A)
 n(A) - np 
 < 0, 01 =
- p \geq 0, 01 = 1 - \P 
\P 

n
n

( \sqrt 
\sqrt )
n  n(A) - np 
n
=
=1-\P \sqrt
< 0, 01 \sqrt


n
\D\xi_1
\D\xi_1

(
)
\int2
 n(A) - np 
1
-u2 /2
<2 =1- \sqrt
= 1 - \P  \sqrt
e
du =
npq 
2\pi
-2

1 - 2Φ(2)  1 - 2 \cdot 0, 47725 = 0, 0455,
где значение Φ(2) интеграла вероятности взято из таблицы. Заметим,
что применение ЦПТ и предельной теоремы Муавра-Лапласа доставляет
лучшую оценку чем ЗБЧ в форме Бернулли (см. ответ в примере 21.7.).
