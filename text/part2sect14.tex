%!TEX root = ../var.tex
\begin{zam}
\label{zam:14.1}
Пусть $\xi : \Omega \to \mathbb{R}$ -- абсолютно непрерывная случайная величина, имеющая плотность $f_\xi (x)$. Построим с помощью функции $g : \mathbb{R} \to \mathbb{R}$ новую случайную величину по формуле $\eta = g(\xi)$. 

Требуется найти функцию распределения и плотность случайной величины $\eta$. Мы решим эту задачу сначала в предположении, что функция $y = g(x)$ дифференцируема и монотонна, т.е. когда во всех точках $x \in \mathbb{R}$ выполнено либо $g′(x) > 0$, либо $g′(x) < 0$.	
\end{zam}
\begin{theorem}
	\label{th:14.2}
Если $\xi$ -- абсолютно непрерывная случайная величина, имеющая функцию распределения $F_\xi (x)$ и плотность $f_\xi (x)$, и если $g :\mathbb{R} \to \mathbb{R}$ -- дифференцируемая и монотонная функция, то случайная величина $\eta = g(\xi)$ имеет плотность вероятности
\begin{equation*}
	f_\eta(y)=f_\xi\left(g^{-1}(y)\right)\left|\diff{[g^{-1}(y)]}{y}\right|
\end{equation*}
\end{theorem}
\begin{proof}
Заметим, что если $g :\mathbb{R} \to \mathbb{R}$  -- монотонная функция, то существует её функция$ g^{-1} :\mathbb{R} \to R$, и выполнено тождество
\begin{equation*}
	g(g^{-1}(y))\equiv y 	
\end{equation*} 
Дифференцируя его, получим тождество
\begin{equation*}
	g'(g^{-1}(y)) \cdot (g^{-1} (y))' \equiv 1,
\end{equation*}
которое означает, что производные $g′$ и $(g^{-1})'$ --
одного знака, т.е. функции $g$ и $g^{-1}$ либо обе возрастающие, либо обе убывающие.


1) Пусть сначала $g$ -- возрастающая функция, т.е. $g' > 0$ и $g^{-1} > 0$. Это означает, что неравенство $g(\xi) \leq y$ можно записать в виде $ \xi \leq g^{-1}(y)$.

\begin{gather*}
	F_{\eta}(y) = F_{g(\xi)}(y) = \P(g(\xi) \leq y) = \P(\xi \leq g^{-1} (y)) = F_\xi (g^{-1} (y)) =\\=
	\int\limits^{g^{-1}(y)}_{-\infty} f_\xi(t) dt=
	%
	\left[
	\begin{aligned}
		t = g^{-1} (\tau ),\qquad & dt = (g^{-1}(\tau ))' d\tau\\
		t = -\infty \mapsto \tau = -\infty,\quad & t = g^{-1}(y) \mapsto \tau = y
	\end{aligned}\right]=\\=
	\int\limits_{-\infty}^y (g^{-1}(\tau))' f_\xi(g^{-1}(\tau)) d\tau.
\end{gather*}



2) Пусть теперь $g$ -- убывающая функция, т.е. $g' < 0 $и $g^{-1} < 0$, тогда
неравенство $g(\xi) \leq y$ можно записать в виде $\xi \geq g^{-1} (y)$.

\begin{gather*}
	F_{\eta}(y) = F_{g(\xi)}(y) = \P(g(\xi) \leq y) = \P(\xi \geq g^{-1} (y)) =\\=
	\int\limits_{g^{-1}(y)}^{\infty} f_\xi(t) dt=
	%
	\left[
	\begin{aligned}
		t = g^{-1} (\tau ),\qquad & dt = (g^{-1}(\tau ))' d\tau\\
		t = -\infty \mapsto \tau = -\infty,\quad & t = g^{-1}(y) \mapsto \tau = y
	\end{aligned}\right]=\\=\int\limits^{-\infty}_y (g^{-1}(\tau))' f_\xi(g^{-1}(\tau)) d\tau=
	\int\limits_{-\infty}^y \left|(g^{-1}(\tau))'\right| f_\xi(g^{-1}(\tau)) d\tau.
\end{gather*}

Объединяя оба случая в один, получим требуемую формулу.
\end{proof}

\begin{definition}
\label{def:14.3}
Пусть $A$ – подмножество на прямой $\mathbb{R}$, т.е. $A \subset
\mathbb{R}$. Функция $\mathbf{1}_A :\mathbb{R} \to \{0, 1\}$, определённая по формуле
\begin{equation*}
	\mathbf{1}_A (x)=\left\{
	\begin{aligned}
		1, \text{ если } x\in A\\
		0, \text{ если } x\notin A
	\end{aligned}
	\right.
\end{equation*}
называется выделяющей функцией множества $A$ или индикатором $A$.
Если $A = [0, \infty)$, то обозначение $\mathbf{1}_A(x)$ сокращают до $\mathbf{1}(x)$. Заметим, что $\mathbf{1}(x) = u(x)$, т.е. совпадает с функцией Хевисайда.	
\end{definition}

\begin{zam}
\label{zam:14.4}
Пусть теперь $g : \mathbb{R} \to \mathbb{R}$, $y = g(x)$ -- дифференцируемая, кусочно монотонная функция, имеющая интервалы монотонности:
\begin{equation*}
	\mathcal{D}=\{D1=(-\infty, a_1], D_2 = (a_1 , a_2 ], \ldots , D_n = (a_{n-1} , \infty)\} 
\end{equation*}

Ясно, что все ограничения $g|_{D_i} : D_i \to \mathbb{R}$, определённые по формулам $g|_{D_i}(x)=g(x)$ являются взаимно однозначными функциями и поэтому имеют обратные $\left(g|_{D_i}\right)^{-1}(y)={g|_{D_i}}^{-1}(y)$ с областями определения $g(D_i)$ соответственно.
\end{zam}

\begin{theorem}[Без доказательства]
	\label{th:14.5}
Если $\xi $ -- абсолютно непрерывная случайная величина, имеющая функцию распределения $F_\xi (x)$ и плотность $f\xi (x)$, и если $g : \mathbb{R} \to \mathbb{R}$ -- кусочно дифференцируемая и кусочно монотонная функция на интервалах $\mathcal{D}$, то случайная величина $\eta = g(\xi)$ имеет плотность вероятности
\begin{equation*}
	f_\eta(y)=\sum\limits_{i=1}^n f_\xi\left(
		{g|_{D_i}}^{-1}(y)
	\right)\cdot
	\left|
		\diff{\left[ {g|_{D_i}}^{-1}(y) \right]}{y}\cdot\mathbf{1}_{g(D_i)}(y)
	\right|
\end{equation*}

\end{theorem}
