%!TEX root = ../var.tex

% Шрифты, кодировки, символьные таблицы, переносы
\usepackage{cmap}
\usepackage[T2A]{fontenc}
\usepackage[utf8x]{inputenc}
\usepackage[english, russian]{babel}

% Это пакет -- хитрый пакет, он нужен но не нужен
\usepackage[mode=buildnew]{standalone}

\usepackage
	{
		% Дополнения Американского математического общества (AMS)
		amssymb,
		amsfonts,
		amsmath,
		% amsthm,
		% misccorr,
		% 
		% Графики и рисунки
		wrapfig,
		graphicx,
		% subcaption,
		float,
		% tikz,
		% tikz-3dplot,
		% caption,
		% csvsimple,
		color,
		% booktabs,
		% pgfplots,
		% pgfplotstable,
		geometry,
		% 
		% Таблицы, списки
		makecell,
		multirow,
		indentfirst,
		%
		% Интегралы и прочие обозначения
		ulem,
		esint,
		esdiff,
		% 
		% Колонтитулы
		fancyhdr,
	}  


% Обводка текста в TikZ
% \usepackage[outline]{contour}

% Увеличенный межстрочный интервал, французские пробелы
% \linespread{1.3} 
\frenchspacing 

% Среднее <#1>

% const прямым шрифтом
\newcommand\ct[1]{\text{\rmfamily\upshape #1}}
\newcommand*{\const}{\ct{const}}

%%%%%%%%%%%%%%%%%%%%%%%%%%%%%%%%%%%%%%%%%%%%%%%%%

\usepackage{tocloft}
\renewcommand{\cftsecleader}{\cftdotfill{\cftdotsep}}
\usepackage{secdot}
\sectiondot{subsection}
\usepackage{setspace}
\usepackage{amsmath, amssymb}
\usepackage{amsthm}