%!TEX root = ../var.tex

% Шрифты, кодировки, символьные таблицы, переносы
\usepackage{cmap}
\usepackage[T2A]{fontenc}
\usepackage[utf8x]{inputenc}
\usepackage[english, russian]{babel}

% Это пакет -- хитрый пакет, он нужен но не нужен
\usepackage[mode=buildnew]{standalone}

\usepackage
	{
		% Дополнения Американского математического общества (AMS)
		amssymb,
		amsfonts,
		amsmath,
		% amsthm,
		% misccorr,
		% 
		% Графики и рисунки
		wrapfig,
		graphicx,
		% subcaption,
		float,
		% tikz,
		% tikz-3dplot,
		% caption,
		% csvsimple,
		color,
		% booktabs,
		% pgfplots,
		% pgfplotstable,
		geometry,
		% 
		% Таблицы, списки
		makecell,
		multirow,
		indentfirst,
		%
		% Интегралы и прочие обозначения
		ulem,
		esint,
		esdiff,
		% 
		% Колонтитулы
		fancyhdr,
	}  


% Обводка текста в TikZ
% \usepackage[outline]{contour}

% Увеличенный межстрочный интервал, французские пробелы
% \linespread{1.3} 
\frenchspacing 

% Среднее <#1>

% const прямым шрифтом
\newcommand\ct[1]{\text{\rmfamily\upshape #1}}
\newcommand*{\const}{\ct{const}}

%%%%%%%%%%%%%%%%%%%%%%%%%%%%%%%%%%%%%%%%%%%%%%%%%

\usepackage{tocloft}
\renewcommand{\cftsecleader}{\cftdotfill{\cftdotsep}}
\usepackage{secdot}
\sectiondot{subsection}
\usepackage{setspace}
\usepackage{amsmath, amssymb}
\usepackage{amsthm}

\usepackage{tabu}
\geometry		
	{
		left			=	2.5cm,
		right 			=	1.5cm,
		top 			=	2cm,
		bottom 			=	3.5cm,
		bindingoffset	=	0cm
	}


\newtheorem{mydef}{Определение}[section]
\newtheorem{theorem}{Теорема}[section]
\newtheorem{lemma}[theorem]{Лемма}%[section]

\makeatletter
% \def\rep@title{1}
\newtheorem*{rep@theorem}{\rep@title}%[section]
\theoremstyle{definition}
\newtheorem*{rep@lemma}{\rep@title}%[section]
\newtheorem*{rep@definition}{\rep@title}%[section]

\newcommand{\newreptheorem}[2]{%
	\newenvironment{rep#1}[1]{%
	 	\def\rep@title{##1}%
	 	\begin{rep@#1}%
	 }{%
	 	\end{rep@#1}
	 }%
 }
\makeatother

\newreptheorem{theorem}{}

\theoremstyle{definition}

\newreptheorem{lemma}{}
\newreptheorem{definition}{}

\newtheorem{consq}[theorem]{Следствие}%[section]
\newtheorem{definition}[theorem]{Определение}%[section]
\newtheorem{deflemma}[theorem]{Лемма-определение}%[section]
\newtheorem{zam}[theorem]{Замечание}
\newtheorem{num}[theorem]{Задача}
\newtheorem{example}[theorem]{Пример}
\newtheorem{prop}[theorem]{Свойство}

\theoremstyle{remark}
\newtheorem{remark}[theorem]{Замечание}%[section]

\fancyfoot{} 
\fancyfoot[C]{\thepage} 

\renewcommand{\backslash}{\smallsetminus}
\newcommand{\ssm}{\smallsetminus}
\newcommand{\noo}{\varnothing}
\newcommand{\oxi}{\overline{\xi}}
\renewcommand{\O}{\noo}

\DeclareMathOperator{\sinc}{sinc}
\newcommand{\M}{\mathsf{M}}
\newcommand{\D}{\mathsf{D}}
\renewcommand{\P}{\mathsf{P}}
\renewcommand{\vec}{\mathbf}
\renewcommand{\Re}{\operatorname{Re}}
\renewcommand{\Im}{\operatorname{Im}}
\newcommand{\cov}{\operatorname{cov}}
\renewcommand{\phi}{\varphi}
\renewcommand{\kappa}{\varkappa}
\renewcommand{\epsilon}{\varepsilon}
\renewcommand{\hat}{\widehat}
\newcommand{\mean}[1]{\langle#1\rangle}
\renewcommand{\contentsname}{Оглавление}