%!TEX root = ../var.tex
\begin{remark}\label{rem:21.1}
Напомним, что по опред. 11.1 случайная величина
есть функция $ \xi : \omega \to R$. Поэтому последовательность случайных величин
$$\xi_1, \xi_2,\dots, \xi_n,\dots = \{\xi_n\}_{n=1}^{\infty}\equiv\{\xi_n\}$$
есть на самом деле последовательность функций $\{\xi_n : \omega \to R\}_{n=1}^{infty}$, определённая на одном и том же пространстве элементарных событий $\omega$.
\end{remark}
В математическом анализе мы изучили следующие две сходимости по-
следовательности функций.

\begin{definition}
	Последовательность функций $\xi_1, \xi_2,\dots , \xi_n,\dots$ сходится к функции $\xi$ поточечно, если для любой точки $\omega \in \omega$ последовательность чисел (значений) $\xi_1(\omega), \xi_2(\omega),\dots, \xi_n(\omega),\dots$ сходится к значению $\xi(\omega)$ функции $\xi$; или короче\footnote{Используя определение сходимости числовой последовательности.} $\lim\limits{n\to\infty}\xi_n(\omega) = \xi(\omega)$
	для любой точки $\omega \in \omega$.
\end{definition}

\begin{definition}
	Последовательность функций $\xi_1, \xi_2,\dots, \xi_n,\dots$ сходится к функции $\xi$ поточечно почти всюду в $\omega$, если подмножество $A \subset
\omega$, в которых поточечная сходимость не выполняется, имеет меру 0, т.е.
$\mu(A) = 0$.
В теории вероятностей объекты $\omega, A, \omega$ являются событиями, а мерой
их наступления является вероятность, поэтому в теории вероятностей
сходимости <<почти всюду>> соответствует так называемая сходимость
<<почти наверное.>>
\end{definition}

\begin{definition}
	Говорят, что последовательность случайных величин $\{\xi_n\}$ сходится почти наверное к случайной величине $\xi$, если имеют
место эквивалентные друг другу равенства
\begin{gather*}
	P\left\{\omega\in\Omega| \lim\limits_{n\to\infty}\xi_n(\omega)=\xi(\omega) \right\}=1 \\
	\text{и} \\
	P\left\{\omega\in\Omega| \lim\limits_{n\to\infty}\xi_n(\omega)\neq\xi(\omega) \right\}=0
\end{gather*}
\end{definition}
\begin{remark}\label{rem:21.5}
Чтобы пользоваться на практике сходимостью <<почти наверное>> необходимо знать, как устроены отображения $\omega \mapsto \xi_n(\omega)$.
Как правило в задачах теории вероятностей известны не сами случайные
величины $\xi_n$, а их функции распределения, скажем $P(\xi_n \leqslant x) = F_{\xi_n}(x) =\int\limits_{-\infty}{x}f_{\xi_n}(t)dt$. Можно ли в таком случае, обладая информацией только о функциях распределения, каким-нибудь образом исследовать сходимость последовательности случайных величин $\{\xi_n\}$? Ответ: да, можно, если потребовать, чтобы вероятность тех элементарных исходов $\omega$, для которыхз начение $\xi_n(\omega)$ не попадает в <<$\epsilon$-окрестность>> числа $\xi(\omega)$, сходилась к нулю при $n \to \infty$. В теории вероятностей эту идею реализуют с помощью так называемой <<сходимости по вероятности>>.
\end{remark}

\begin{definition}
	Говорят, что последовательность случайных величин $\{\xi_n\}$ сходится по вероятности к случайной величине $\xi$, если для любого
$\varepsilon \in (0,\infty)$ имеют место эквивалентные равенства
\begin{gather*}
\label{def:21.6}
	\lim_{n\to\infty}P(|\xi_n- \xi|\geqslant \varepsilon)=0 \\
	\lim_{n\to\infty}P(|\xi_n- \xi|\leqslant \varepsilon)=1 \tag{*}
\end{gather*}

\end{definition}
Если ${\xi_n}$ сходится по вероятности к $\xi$, то вместо пределов (\ref{def:21.6}) коротко пишут <<$\xi_n\stackrel{P}{\to} \xi$ при $n \to \infty,$>> а т.к. $n \to \infty$ всегда, то ещё короче: "$\xi_n\stackrel{P}{\to} \xi$.
Хотя на практике для проверки такой сходимости проверяют выполнение
одного из равенств (\ref{def:21.6}).
\begin{example}
Рассмотрим последовательность случайных величин $\{\xi_n\}$ в которой для каждого n случайная величина задана следующим рядом распределения
\begin{center}
	\begin{tabular}{|c|c|c|}
	\hline
	$\xi_n$ & $c$& $n^3$\\ \hline
	$P$ & $1-1/n$& $1/n$\\ \hline
	\end{tabular}	
\end{center}

	Докажем, что эта последовательность сходится по вероятности к вырожденной случайной (детерминированной) величине $\xi$, имеющей ряд распределения
	\begin{tabular}{|c|c|}
	\hline
	$\xi$ & $0$\\ \hline
	$P$ & $1$\\ \hline
	\end{tabular}, а проще говоря, сходится по вероятности к нулю.

	Действительно, зафиксируем произвольное $\varepsilon > 0$. Для всех $n$, начиная
с некоторого $n_0$ такого, что $n^3_0>\varepsilon$, выполнено равенство
\begin{equation*}
	P(|\xi_n-0|\geqslant\varepsilon)=P(\xi_n\geqslant\epsilon)=P(\xi_n=n^3)=\frac{1}{n}.
\end{equation*}
Применяя теперь опред. 21.6 получаем
\begin{equation*}
	\lim\limits_{n\to\infty} P(|\xi_n-0|\geqslant\epsilon)=\lim\limits_{n\to\infty}\frac{1}{n}=0
\end{equation*}
т.е. последовательность случайных величин $\xi_1, \xi_2, \dots , \xi_n, \dots $сходится по вероятности к $\xi$, т.е. $\xi_n \stackrel{P}{\to}\xi$.

В нашем примере случайные величины $\xi_n$ с ростом $n$ могут принимать
всё б´oльшие и б´oльшие значения, но с всё меньшей и меньшей вероятностью.
\end{example}


\begin{remark}\label{rem:21.7}
Сходимость по вероятности может не сопровождаться сходимостью математических ожиданий (или других начальных и
центральных моментов): 
из $\xi_n\stackrel{P}{\to} \xi$ не следует, что $\M\xi_n \to \M\xi$. Например,
в предыдущем примере $\xi_n
\stackrel{P}{\to} \xi$, однако $\M\xi_n^2 = n2$ и $\M\xi = 0$, и ясно, что
последовательность $1, 4, 9, \dots$ не сходится к нулю.
Сходимость по вероятности обладает теми же свойствами, как и поточечная сходимость.
\end{remark}

\begin{theorem}[Без доказательства]
	Если $\xi_n \stackrel{P}{\to} \xi$ и $\eta_n\stackrel{P}{\to} \eta$, то
	\begin{enumerate}
		\item  $\xi_n + \eta_n\stackrel{P}{\to} \xi + \eta$,
		\item $\xi_n \cdot \eta_n\stackrel{P}{\to} \xi \cdot \eta$,
		\item Если $g$ — непрерывная функция, то $g(\xi_n) \stackrel{P}{\to} g(\xi)$.
		\item Если $\xi_n \stackrel{P}{\to} C$ и функция g непрерывна в точке C, то $g(\xi_n) \stackrel{P}{\to} g(C)$.
	\end{enumerate}
\end{theorem}

Другой тип сходимости случайных величин, который нам понадобится
в дальнейшем, определяется через поточечную сходимость функций распределений. Это так называемая <<слабая сходимость.>> Пусть задана последовательность случайных величин $\{\xi_n\}_{n=1}^{\infty}$ с функциями распределения $\{F_{\xi_n}(x)\}_{n=1}^{\infty}$, и задана случайная величина $\xi$ с функцией распределения $F_\xi(x)$.

\begin{definition}
	Говорят, что последовательность случайных величин $\{\xi_n\}_{n=1}^{\infty}$ слабо сходится к случайной величине $\xi$, если последовательность их функций распределения $\{F_{\xi_n}(x)\}^{\infty}_{n=1}$ сходится поточечно к функции распределения $F_\xi(x)$, т.е. $\lim\limits{n\to\infty}
	F_{\xi_n}(x) = F_\xi(x)$.

Если $\{\xi_n\}$ слабо сходится к $\xi$, то коротко пишут $\xi_n \Rightarrow \xi$. На практике для проверки слабой сходимости проверяют выполнение равенства
$$\lim \limits{n\to\infty} F_{\xi_n}(x) = F_\xi(x)$$ для любого $x \in \mathbb{R}$.
\end{definition}

\begin{theorem}
	\hspace{0pt}
	\begin{enumerate}
		\item Если $\xi_n\stackrel{P}{\to}\xi,$ то $\xi_n\Rightarrow\xi$
		\item Если $\xi_n\Rightarrow C=const$, то $\xi_n \stackrel{P}{\to} C$
		\item Если $\xi_n\stackrel{P}{\to} C$ и $\eta_n\Rightarrow\eta$, то $\xi_n\cdot\eta_n\Rightarrow C\eta$
		\item Если $\xi_n\stackrel{P}{\to} C$ и $\eta_n\Rightarrow\eta$, то $\xi_n+\eta_n\Rightarrow C+\eta$
	\end{enumerate}
\end{theorem}