%!TEX root = ../var.tex

Возникновение теории вероятностей как науки относят к средним векам, когда появилась возможность и возникла необходимость изучения математическими методами азартных игр (таких как орлянка, кости, рулетка). 

Самые ранние работы учёных в области теории вероятностей относятся к XVII веку. Первоначально её основные понятия не имели строго математического описания. Задачи, из которых позже выросла теория вероятностей представляли набор некоторых эмпирических фактов о свойствах реальных событий, которые формулировались с помощью наглядных описаний. 

Исследуя прогнозирование выигрыша при бросании костей в письмах друг другу, Блез Паскаль и Пьер Ферма открыли первые вероятностные закономерности. Решением тех же задач занимался и Христиан Гюйгенс. При этом с перепиской Паскаля и Ферма он знаком не был и методику решения изобрёл самостоятельно. 

Его статья, в которой он ввёл основные понятия теории вероятностей (понятие вероятности как величину шанса; математическое ожидание для дискретных случаев в виде цены шанса). В своей статье он использует (не сформулированные ещё в явном виде) теоремы сложения и умножения вероятностей. Статья была опубликована в печатном виде на двадцать лет раньше (1657 г.) издания писем Паскаля и Ферма (1679 г.). 

Важный вклад в теорию вероятностей внёс Якоб Бернулли, он дал доказательство закона больших чисел в простейшем случае независимых ис- пытаний. В первой половине XIX века теория вероятностей начинает применяться к анализу ошибок наблюдений; Лаплас и Пуассон доказали первые предельные теоремы. 

Во второй половине XIX века основной вклад внесли русские учёные П. Л. Чебышёв, А. А. Марков и А. М. Ляпунов. В это время были доказаны закон больших чисел, центральная предельная теорема, а также разработана теория цепей Маркова. 

Современный вид теория вероятностей получила благодаря аксиоматике, предложенной Андреем Николаевичем Колмогоровым. В результате теория вероятностей приобрела строгий математический вид и окончательно стала восприниматься как один из разделов математики.
Википедия,
Статья "Теория вероятностей".