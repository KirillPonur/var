%!TEX root = ../var.tex

\subsection{Экспериментальное нахождение вероятности}

В этом пункте описан способ экспериментального определения вероятности
наступления так называемых массовых событий.

\begin{definition}
  	Событие A называется \textit{массовым}, если опыт (эксперимент, испытание), при котором событие A может произойти, можно повторить \textit{неограниченное число раз при одних и тех же условиях}.
\end{definition} 

Методами теории вероятностей изучают (в основном) эксперименты, которые порождают массовые события. Всюду до конца лекций будут рассматриваться только массовые события.

\begin{example}
1) Опыт: однократное подбрасывание ломаного гроша.
События выпадение орла $O$ и выпадение решки $P$ являются массовыми событиями. Заметим, что пространство элементарных событий 
$\Omega = \{O, P \}$ состоит из конечного числа элементарных исходов.

2) Опыт: подбрасывание ломаного гроша до появления первого орла. События $P , OP , OOP, \ldots, O \ldots OP , \ldots$ являются массовым. Заметим, что в этом примере пространство $\Omega=\{P, OP, OOP, \ldots, O \ldots OP, \ldots\}$ состоит из
счётного количества элементарных исходов: положение кольца однозначно определяется положением его центра.

3) Опыт: бросание обручального кольца на клетчатую скатерть (диаметр кольца меньше стороны клетки). Событие $A$: кольцо падает внутрь какой-нибудь клетки, не пересекая её границы. Это событие является массовым.
Заметим, что в этом примере пространство состоит из несчётного количества элементарных исходов.
 \end{example} 

Типичность этих трёх примеров состоит в том, что в теории вероятностей встречаются три типа пространств элементарных событий: конечные, счётные и несчётные.

\begin{definition}
	Если в результате n испытаний массовое событие $A$ произошло $\mu(A)$ раз, то число $\frac{\mu(A)}{n}$
называется \textit{относительной частотой} появления события $A$.
\end{definition}

Для каждого примера 4.2 можно провести $n$ опытов, сосчитать число $\mu(A)$ появлений этого события и подсчитать относительную частоту $\frac{\mu(A)}{n}$.

Массовые события обладают свойством "статистической устойчивости", а именно: \textit{при увеличении числа экспериментов относительная частота $\frac{\mu(A)}{n}$
появления события $A$ имеет тенденцию стабилизироваться, стремясь к некоторому числу $P(A)$.}
\begin{definition}
Число $P(A) = lim_{n\to\infty} \frac{\mu(A)}{n}$ называется \textit{статистической вероятностью} события $A$ и находится при больших $n$ по приближённой формуле $P(A)\approx \mu(A)$.
\end{definition} 

\subsection{Вероятность на конечном пространстве.}
Пусть задано конечное пространство $\Omega=\{\omega_1,\omega_2,\ldots,\omega_n\}$, состоящее из $n$ элементарных событий (исходов), и заданы вероятности наступления этих
событий $P(\omega_1 ) = p_1 , P(\omega_2 ) = p_2 , \ldots , P(\omega_n ) = p_n$ так, что $p_1 + p_2 + \ldots + p_n = 1$.
Обозначим через $\omega$ переменную величину, принимающую значения из $\Omega$.

\begin{definition}
	Такое соответствие записывает в виде таблицы
% omega omega1 omega2 . . . omegan
% ,
% P(ξ) p1 p2 . . . pn
\begin{table}
	% 
\end{table}

которая называется \textit{рядом распределения} случайных исходов или \textit{законом
распределения}. Если $n = 2$, то ряд называется \textit{схемой Бернулли}. Если $n \geq 3$,
то ряд называется \textit{схемой независимых испытаний с несколькими исходами}.
\end{definition}

Ряд распределения задаёт функцию в виде таблицы, которая каждому элементарному исходу ставит в соответствие вероятность его наступления.
Потребуем теперь выполнение акс. $\mathcal{P}3$.

\begin{deflemma}[Формула вероятности на конечном пространстве]
	Если  вероятность, определяемая рядом распределения из опред. 4.5, подчиняется акс. $\mathcal{P}3$, то вероятность события $A = \{\omega_{i_1} , omegai2 , . . . , omegaik \} \subset \Omega$ вычисляется по формуле

$P(A) = p_{i_1} + p_{i_2} + \ldots + p_{i_k} ,$

которая называется формулой вероятности на конечном пространстве.
\end{deflemma}
\begin{proof}
 \begin{gather*}
 	P(A) = P (\omega_{i_1} , \omega_{i_2} , \ldots , \omega_{i_k}) =\\= P \left( \bigcup\limits_{j=1}^k \{\omega_{i_j} \} \right) \stackrel{\mathcal{P}3}{=} \sum_{j=1}^{k} P(\omega_{i_j}) = \sum_{j=1}^{k} p_{i_j} = p_{i_1} + p_{i_2} + \ldots + p_{i_k}.
 \end{gather*}
 	
 \end{proof} 

\begin{example}
Простейший ряд распределения имеет эксперимент с детерминированным исходом, $\Omega = \{ \omega \}$ (см. опред. 2.3):

% ξ omega
% .
% P 1

\begin{table}
	% 
\end{table}

Этот тривиальный случай удовлетворяет всем аксиомам и определениям, однако никакого значения в теории вероятностей не имеет. Придётся с этим
мириться.
\end{example}

\begin{example}
	Пусть в результате опыта могут возникнуть только
два события: <<успех>>, который обозначается единицей — 1 и наступает c вероятностью $p$, и <<неудача>>, которая обозначается нулём — 0 и наступает c
вероятностью $q = 1−p$; $\Omega = \{0, 1\}$. Такой опыт называется \textit{схемой Бернулли}\footnote{
	Якоб Бернулли (Jakob Bernoulli, 1654-1705), швейцарский математик
}
и имеет ряд распределения

\begin{table}
	% 
\end{table}
% ξ
% 0
% 1
% .
% P q =1−p p

\end{example}

Схема Бернулли является первым нетривиальным примером вероятности на конечном пространстве.

Схема Бернулли имеет простую интерпретацию. Рассмотрим ломаный грош с вероятностями выпадения орла $p$ и решки $q = 1 − p$. Обозначим появление орла через 1, а его не выпадение через 0, получим схему Бернулли.

\textbf{Геометрическая интерпретация} (схемы независимых испытаний с несколькими исходами). Рассмотрим произвольный выпуклый многогранника с $n$ гранями, на которых написаны символы $\omega_1 , \omega_2 , \ldots , \omega_n$ ; при этом многогранник должен быть таким, чтобы перпендикуляр, опущенный из его центра тяжести на любую грань, пересекал эту грань в её внутренней точке (чтобы многогранник, падая на эту грань, не перекатывался на другую).

Пусть вероятности выпадения многогранника гранью вниз равны соответственно $p_1 , p_2 , \ldots , p_n$ , где естественно $p_1 + p_2 + \ldots +p_n = 1$. Легко видеть, что вероятность $p_i$ пропорциональна телесному углу $\alpha_i$ с вершиной $C$ в центре тяжести многогранника, опирающегося на грань $\omega_i$ ; а т.к. сумма всех телесных углов с вершиной $C$ равна $4\pi$ стерадиан, т.е. $\alpha_1 + \alpha_2 + \ldots + \alpha_n = 4\pi$, то
$\frac{\alpha_1}{4\pi} + \frac{\alpha_2}{4\pi} + \ldots + \frac{\alpha_n}{4\pi}
 = 1$, поэтому $p_i = 4\pi$ .

\subsection{Классическая вероятность}

Классическая вероятность является частным случаем вероятности на конечном пространстве, когда вероятность подчинена принципу равной вероятности. Исторически классическая вероятность применялась в теории азартных игр и появилась раньше вероятности на конечном пространстве.

\textbf{Принцип равной вероятности.} \textit{Если на конечном пространстве omega вероятность наступления его элементарных исходов $\omega_1 , \omega_2 , \ldots , \omega_n$ одна и та же,
$$p_1 = p_2 = \ldots = p_n = p$$,
то говорят, что вероятность на конечном пространстве удовлетворяет
принципу равной вероятности (или равной возможности).}

Выпадение орла и решки при подбрасывании симметричной монеты; выпадение граней при подбрасывании правильного многогранника (тетраэдра, куба, октаэдра, додекаэдра или икосаэдра); появление к.-л. карты из полной
колоды карт удовлетворяют принципу равной возможности.

Классическая вероятность характеризуется только числом элементарных исходов $n$ в пространстве $\Omega$. Она имеет ряд распределения

\begin{table}
	% 
\end{table}
% omega
% omega1 omega2 . . . omegan
% ,
% P(omega) p p . . . p

где $np = 1$, поэтому $p = \frac{1}{n}$ .


\begin{lemma}[(Формула классической вероятности.)]
Если вероятность $P$ удовлетворяет принципу равной возможности, то вероятность наступления события $A = \{ \omega_{i_1} , \omega_{i_2} , \ldots , \omega_{i_k} \} \subset \Omega$ определяется по формуле, называемой формулой классической вероятности,
$$P(A) = \frac{k}{n}$$
\end{lemma} 

\begin{proof}
	$P(A) = p_{i_1} + p_{i_2} + \ldots + p_{i_k} = kp = \frac{k}{n}$
\end{proof}

Если обозначить число $k$ элементов множества $A$ через $\mu(A)$, а число $n$ элементов пространства $\Omega$ через $\mu(\Omega)$, то формулу классической вероятности
можно записать в виде

$$P(A) = \frac{\mu(A)}{\mu(\Omega)}$$

Число $\mu(A) = k$ называется числом исходов, \textit{благоприятствующих} наступлению события $A$.
\begin{zam}

1) Вероятности элементарных исходов в экспериментах
с шариками, описанные в теоремах 1.4, 1.6 и 1.7, удовлетворяют принципу равной возможности. Эти вероятности соответственно $1/A_n^k , 1/C_n^k и 1/n^k$ .

2) Вероятности элементарных исходов в теор. 1.9 (выбор с возвращением и без учёта порядка) не удовлетворяют этому принципу. Например, при $n = 2$
и $k = 2$ вероятности элементарных событий имеют ряд распределения

% omega
% (1, 1) (1, 2) (2, 2)
% .
% P(omega) 1/4
% 1/2
% 1/4
\begin{table}
	% 
\end{table}

3) Следует отметить, что при рассмотрении подобных вопросов ошибались даже такие великие математики, как, например, Д’Аламбер. Так, однажды у Даламбера спросили, с какой вероятностью монета, брошенная дважды, хотя бы один раз выпадет гербом. Ответ учёного был $\frac{2}{3}$ , т.к. он считал, что есть 3 возможных исхода (герб-герб, герб-решка, решка-решка) и среди них 2 благоприятствующих. Д’Аламбер пренебрегал тем, что эти три возможных исхода не равновозможны. Правильным ответом является $\frac{3}{4}$ , поскольку из четырёх равновозможных исходов (герб-герб, герб-решка, решкагерб, решка-решка) три благоприятствуют указанному событию. Точка зрения Д’Аламбера была даже опубликована во Французской энциклопедии в 1754 г. в статье "Герб и решётка"("Croix on pile").
\end{zam}
 

\subsection{Вероятность на счётном пространстве}
Пусть теперь $\Omega = \{\omega_1 , \omega_2 , \ldots , \omega_n , \ldots \}$ – счётное пространство. Пусть $p_1 + p_2 + \ldots + p_n + \ldots = 1$ — сходящийся (к единице) числовой ряд, удовлетворяющий
для всех $i \in N$ условию $0 < p_i < 1$. Последнее условие позволяет трактовать члены этого ряда как вероятности элементарных событий пространства $\Omega$.

\begin{definition}
	Вероятность элементарных исходов, заданная в виде таблицы

% omega
% omega1 omega2 . . .
% P(omega) p1 p2 . . .

% omegai . . .
% ,
% pi . . .
\begin{table}
	% 
\end{table}

называется рядом распределения на счётном пространстве.
Потребуем теперь выполнение акс. $\mathcal{P}3$.
\end{definition}

\begin{lemma}[(Формула вероятности на счётном пространстве.)]

Если события из пространства $\Omega$ подчиняются акс. $\mathcal{P}3$, то вероятность события $A = \{\omega_{i_1} , \omega_{i_2} , \ldots , \omega_{i_k} , \ldots \} \subset \Omega$ вычисляется по формуле
$P (A) = p_{i_1} + p_{i_2} + \ldots + p_{i_k} + \ldots .$
\end{lemma}

\begin{proof}
\begin{gather*}
	P(A) = P(\omega_{i_1},\omega_{i_2}, \ldots, \omega_{i_k}, \ldots) =\\= P \left( \bigcup_{j=1}^\infty \{ \omega_{i_j}\} \right) \stackrel{\mathcal{P}3}{=} \sum_{j=1}^{\infty} P(\omega_{i_j}) = \sum_{j=1}^{\infty} p_{i_j} = p_{i_1} + p_{i_2} + \ldots + p_{i_k} + \ldots 
\end{gather*}
	
\end{proof}

\begin{zam}
Каждому степенному ряду на той части его интервала
сходимости, на которой его члены положительны, можно поставить в соответствие параметрическое семейство распределений со счётным пространством.
Рассмотрим, например, экспоненту $e^\lambda$ . Её ряд МакЛорена
$1+ \frac{\lambda}{1!} + \frac{\lambda^2}{2!} + \ldots + \frac{\lambda^k}{k!} = e^\lambda$
сходится при всех значениях $−\infty < \lambda < \infty$ и имеет положительные члены при $\lambda > 0$. Умножим обе части этого тождества на $e^{−\lambda}$ , получим тождество

$$e^{−\lambda}+ \frac{\lambda}{1!} e^{−\lambda} + \frac{\lambda^2}{2!} e^{−\lambda}+ \ldots + \frac{\lambda^k}{k!} e^{−\lambda}= 1$$

Составим следующий ряд распределения:
\begin{table}
	% 
\end{table}
% k
% 0
% P\lambda (k) e−\lambda

% 1
% \lambda −\lambda
% 1! e

% 2
% ...
% \lambda −\lambda
% ...
% 2! e
% 2

% 19

% k
% ...
% .
% \lambda −\lambda
% ...
% k! e
% k
\end{zam}

\begin{definition}
Распределение, реализуемое этим рядом, называется \textit{распределением Пуассона с параметром $\lambda$.}

На рис. 25 показаны функции $P_\lambda (k)$ для для параметра $\lambda = 0, 1; 1; 10,$ где
пунктиром показаны огибающие. В каждом случае сумма длин вертикальных отрезков равна 1.

Распределение Пуассона описывает, например, вероятность $P_\lambda (k) = \frac{\lambda^k e^{-\lambda}}{k!}$
поступления на телефонную станцию $k$ звонков за какой-нибудь фиксированный промежуток времени, где число звонков $k = 0, 1, 2, \ldots .$
\end{definition}. 

\begin{zam}
Другим источником построения распределений со счётным пространством являются сходящиеся числовые ряды с положительными членами. Например, известно (Л. Эйлер), что
$$\frac{1}{1^2} + \frac{1}{2^2} + \frac{1}{3^2} + \ldots + \frac{1}{k^2} + \ldots = \frac{\pi^2}{6}.$$
Разделив обе части этого числового тождества на $\frac{\pi^2}{6}$ , можно получить (безымянный) ряд распределения, задаваемый формулой $P(k) = \frac{1}{k^2} * \frac{6}{\pi^2}$ , со счётным пространством $\Omega = \mathcal{N}$.
Среди числовых рядов наиболее встречающимися в теории вероятностей являются геометрические прогрессии с положительными членами.
\end{zam}

\begin{definition}
Если члены ряда $p_1 + p_2 + \ldots + p_n + \ldots = 1$ положительны и являются членами убывающей геометрической прогрессией с первым членом $p_1 = p$ и с знаменателем $q = 1 − p$, то вероятность на счётном пространстве называется геометрическим распределением. Геометрическое распределение имеет ряд

\begin{table}
	% 
\end{table}
% τ1 omega1 omega2 . . .
% P p qp . . .

% q

% omegak
% k−1

% ...
% .
% p ...
\end{definition}

\begin{example}
Рассмотрим схему Бернулли (ломаного гроша) с вероятностью "успеха (выпадения орла = 1)" равной $p$. Она имеет ряд распределения

\begin{table}
	% 
\end{table}
% ξ
% 0
% 1
% .
% P 1−p p

Пусть эксперимент состоит в том, что испытания по схеме Бернулли проводятся неограниченное число раз. С таким экспериментом связаны две случайные величины: $\tau_1$ — номер первого выпавшего орла и $\tau_0$ — число выпавших решек, появившихся до первого орла. Легко подсчитать, что случайные величины $\tau_0$ и $\tau_1$ имеют ряды распределения

\begin{table}
	% 
\end{table}
% τ0 0 1 . . .
% P p qp . . .

% k ...
% qk p . . .

% и
% 20

% τ1 1 2 . . .
% P p qp . . .

% k
% q

% k−1

% ...
% .
% p ...


В строке вероятностей эти ряды содержат одну и ту же геометрическую
прогрессию; они называются соответственно $\tau_0$ -- и $\tau_1$ -- геометрическим распределениями. Эти случайные величины связаны очевидной формулой $\tau_1 = \tau_0 + 1$.
Формула вероятность для $\tau_0$ -геометрического распределения определена по формуле
$$P(\tau_0 = k) = q^k p,$$
где $0 < p < 1$ и $q = p − 1$.
Формула вероятность для $\tau_1$ -геометрического распределения:
$$P(\tau_1 = k) = q^{k−1} p,$$
где $0 < p < 1$ и $q = p − 1.$
\end{example}

\subsection{Геометрическая вероятность}

Рассмотрим ограниченную измеримую область $\Omega \subset \mathcal{R}^n$ , состоящую из несчётного множества точек. Измеримость области означает: на прямой область $\Omega \subset \mathcal{R}^1$ имеет конечную ненулевую длину, на плоскости область $\Omega \subset \mathcal{R}^2$ имеет конечную ненулевую площадь, в пространстве область $\Omega \subset \mathcal{R}^3$ имеет конечный ненулевой объём и т.д. Пусть на $\Omega$ определена $\sigma$-алгебра $\mathfrak{A}$. Для любого события $A \in \mathfrak{A}$ обозначим через $\mu(A)$ его меру (длину, площадь, объём и т.д. соответственно). Пусть эксперимент состоит в том, что в область $\Omega$ бросают точку.
\textbf{Принцип равномерности}\footnote{
Вероятности, которые не подчиняются этому принципу, являются главным предметом изучения теории вероятностей. Они будут изучены в гл. 2. <<Теория случайных величин>>.	
}. 
\textit{Если для любого события$A \in \mathfrak{A}$ его вероятность задаётся по формуле
$P(A) = \alpha \mu(A),$
где $\alpha$ — постоянное число, независящее от выбора события $A$ (т.е. не зависящее от формы $A$ и его расположения в $\Omega$), то говорят, что вероятность
$P(A)$ удовлетворяет принципу равномерности.}

\begin{lemma}

 Если вероятность $P(A)$ удовлетворяет принципу равномерности, то

$$\alpha=\frac{1}{\mu(\Omega)}$$
\end{lemma}

\begin{proof}
	Подставим в формулу $P(A) = \alpha \mu(A)$ достоверное событие $\Omega \in \mathfrak{A}$, получим: $1 = \alpha\mu(\Omega).$
\end{proof} 

\begin{definition}
Полученная формула
$$P(A) = \frac{\mu(A)}{\mu(\Omega)}$$
называется формулой \textit{геометрической} вероятности. (Не путать с геометрическим рядом распределения!)
\end{definition}

Рис. 3: Задача Бюффона.

Рис. 4: Пространство элементарных событий omega в задаче Бюффона.

\begin{example}[задача Бюффона (1777 г.)]
На плоскости нарисовано счётное множество параллельных прямых. Расстояние между соседними прямыми равно $a$. На плоскость брошена игла длины $l < a$. Какова вероятность
того, что игла пересечёт одну из прямых?
\textit{Решение}. Возможные положения иглы на плоскости полностью определяются двумя координатами: расстоянием $x$ от середины иглы до ближайшей
прямой и острым углом $\phi$ между иглой и перпендикуляром к параллельным прямым, см. рис. 3. Ясно, что $x \in [0, \frac{a}{2} ]$ и $\phi \in [0, \frac{\pi}{2} ]$. Поэтому множество $\Omega = [0, \frac{\pi}{2}] * [0, \frac{a}{2} ]$ есть пространство элементарных исходов этого эксперимента, см. рис. 4, и $\mu(\Omega) = \frac{\pi a}{4}
.$
Событие $A =$ {игла пересечёт одну из прямых} эквивалентно неравенству
$A = \{x \leq 2{{l}} \cos \phi \}$, поэтому множество благоприятных исходов располагается в пространстве $\Omega$ под графиком $x = \frac{ {{l}}}{2} \cos \phi$. Вычислим

$$\int_{0}^{\frac{\pi}{2}} \frac{{{l}}}{2} \cos \phi d\phi = \frac{{{l}}}{2} \sin \phi \bigg|_0^{\frac{pi}{2}} = \frac{{{l}}}{2}$$
Отсюда получаем $P(A) = \frac{\mu(A)}{\mu(\Omega)}
 = \frac{2{{l}}}{\pi a}$
\end{example}
\begin{example}["Парадокс" Бертрана\footnote{
	Жозеф Луи Франсуа Бертран (Joseph Louis Francois Bertrand, 1822 — 1900), французский математик.
} (1888 г.)]
	
В круге наудачу выбирается хорда. Какова вероятность того, что её длина будет больше, чем длина стороны вписанного в круг правильного треугольника?
\textit{Решение}. Обозначим через $A$ событие, состоящее в том, что длина хорды будет больше, чем длина стороны вписанного в круг правильного треугольника. Существует по крайней мере три способа "выбрать наудачу" хорду в
круге.

Рис. 5: Парадокс Бертрана.

1-й способ. Зафиксируем один конец хорды O на окружности. Положение другого конца хорды M будем считать равномерно распределённым на окружности, см. рис. 5.1. Пусть координата конца $M$ хорды есть длина дуги $OM$ окружности, проходимой против часовой стрелки, тогда пространство элементарных событий $\Omega$ есть окружность, т.е. $\mu(\Omega) = 2 \pi R$. Благоприятным
исходом является положение конца хорды на дуге $BC$. Событие $A$ всех благоприятных исходов есть дуга $BC$, которая составляет третью часть окружности, т.е. $\mu(A) = \frac{2 \pi}{3} R$. Поэтому $P(A) = \mu(\Omega) = \frac{1}{3}$.

2-й способ. Для каждой точки $M$ внутри круга (кроме его центра\footnote{
	Т.к. вероятность попадания точки в центр круга равна нулю, то наступление этого события не влияет
на вероятность какого-либо события.
}) 
существует единственная хорда, для которой точка $M$ является её серединой, см. рис. 5.2. Поэтому, бросая точку $M$ в круг радиуса $R$, можно по ней однозначно восстановить хорду. Середину $M$ хорды будем считать равномерно распределённой в круге. Пространство элементарных событий $\Omega$ есть круг
радиуса $R$, его площадь $\mu(\Omega) = \pi R^2$ . Благоприятными событию $A$ являются положения середины $M$ хорды внутри окружности, вписанной в треугольник. Легко подсчитать, что радиус вписанной окружности равен $\frac{R}{2}$ . Поэтому $\mu(A) = \frac{\pi R^2}{4}$ и $P(A) = \frac{\mu(A)}{\mu(\Omega)} = \frac{1}{4}$.

3-й способ. Можно ограничиться рассмотрением только хорд, которые перпендикулярны какому-нибудь диаметру, например $C$ (остальные положения могут быть получены поворотом), см. рис. 5.3. Середину хорды будем считать равномерно распределённой на диаметре $BC$. Пространство элементарных событий omega есть диаметр $BC$, которому перпендикулярны хорды, его длина $\mu(\Omega) = 2R$. Благоприятными событию $A$ являются положения середины хорды на отрезке $DE$, лежащем на диаметре $BC$, так что центр отрезка $DE$ совпадает с центром круга. Можно видеть, что длина отрезка $DE$ есть $\mu(A) = R$. Поэтому $P(A) = \frac{\mu(A)}{\mu(\Omega)} = \frac{1}{2}$.
Причина разных ответов заключается в том, что условие в круге наудачу выбирается хорда определяет эксперимент не однозначно. В решении мы провели три разных эксперимента по выбору хорды, и поэтому каждом случае был получен правильной ответ.

\end{example}

% \end{document}