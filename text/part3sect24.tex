%!TEX root = ../var.tex

\begin{zam}
\label{zam:24.1}
Решение многих задач теории вероятностей, особенно тех, которые связаны с суммированием независимых случайных величин, удаётся получить с помощью т.н. характеристических функций.
Связь между преобразованием Фурье и характеристической функцией можно описать следующим образом. Если обозначить преобразование Фурье через $\Phi$, то формулы прямого и обратного преобразования можно записать
следующим образом
\begin{gather*}
	\phi(t) = \Phi[f] = \frac{1}{\sqrt{2\pi}}\int\limits_{-\infty}^{\infty}e^{-ixt}f(x)dx, \\
%и
	f(x) = \Phi^{-1}[\phi] = \frac{1}{\sqrt{2\pi}} \int\limits_{-\infty}^{\infty}e^{itx}\phi(t)dt.   
\end{gather*}

Если $\xi$ — случайная величина, имеющая плотность $f_{\xi}(x)$, то в терминах преобразования Фурье её характеристическая функция есть
$$\Theta_{\xi} (t) = \sqrt{2\pi}\Phi^{-1}[f_{\xi}(x)] = \int\limits_{-\infty}^{\infty}e^{itx}dx$$
\end{zam}

\begin{definition}
\label{def:24.2}
	Пусть $f_{\xi}(x)$ -- плотность случайной величины $\xi$,
тогда функция
$$\Theta_{\xi}(t) = \M e^{it\xi} = \int\limits_{-\infty}^{\infty} e^{itx}f_{\xi}(x)dx$$

называется характеристической функцией случайной величины $\xi$.
\end{definition}

\begin{zam}\-
\label{zam:24.3}

1) Пусть $\xi$ и $\eta$ -- вещесвенные случайные величины. Составим комплексную случайную величину $\xi + i\eta$. Мы распространяем действие знака математического ожидания $\M$ на любую комплексную случайную величину по свойству линейности:
$$\M(\xi + i\eta) = \M\xi + i \M\eta.$$

2) Зная характеристическую функцию $\Theta_{\xi} (t)$, можно однозначно восстановить функцию распределения, а также плотность вероятности или ряд распределения случайной величины $\xi$. Например, если модуль характеристической функции $\Theta_{\xi} (t)$ интегрируем на всей прямой, то плотность $f_{\xi}(x)$
случайной величины $\xi$ находится по формуле
$$f_{\xi}(x) =\frac{1}{2\pi}\int\limits_{-\infty}^{\infty}e^{itx}\Theta_{\xi} (t) dt.$$
\end{zam}

\textbf{Свойства характеристических функций}

\begin{theorem}[Без доказательства]
\label{th:24.4}
	Характеристическая функция $\Theta_{\xi} (t)$ любой случайной величины $\xi$ равномерно непрерывна.
\end{theorem}

\begin{lemma}\-
\label{lemma:24.5}

\begin{enumerate}
	\item $\Theta_{\xi} (0) = 1$
	\item $|\Theta_{\xi} (t)| \leq 1 \text{для} −\infty < t < \infty$.
\end{enumerate}
\end{lemma}

\begin{proof}\-

\begin{enumerate}
	\item $\Theta_{\xi} (0) = \M e^0 = 1.$
	\item $|\Theta_{\xi} (t)| = |\M e^{it\xi} | \leq \M|e^{it\xi} | = \M 1 = 1.$
\end{enumerate}
\end{proof}

 \begin{lemma}
 \label{lemma:24.6}
  	Если $\eta = a\xi + b$, где $a$ и $b$ — постоянные, то
$$\Theta_{\eta} (t) = \Theta_{\xi} (at) e^{ibt}.$$
  \end{lemma} 

  \begin{proof}
  	$\Theta_{\eta} (t) = \M e^{it\eta} = \M e^{it(a\xi+b)} = e^{ibt} \M e^{iat\xi} = e^{ibt} \Theta_{\xi} (at)$
  \end{proof}

\begin{lemma}
\label{lemma:24.7}
 Если $\xi_1$ и $\xi_2$ -- независимые случайные величины, то
$$\Theta_{\xi_1 +\xi_2} (t) = \Theta_{\xi_1} (t) \cdot \Theta_{\xi_2} (t).$$
 \end{lemma} 

\begin{proof}
Т.к. $\xi_1$ и $\xi_2$ -- независимые величины, то $e^{it\xi_1}$ и $e^{it\xi_2}$ тоже независимы. Тогда
$$\Theta_{\xi_1 +\xi_2} (t) = \M e^{it(\xi_1 +\xi_2 )} = \M e^{it\xi_1} \cdot e^{it\xi_2} \stackrel{16.5.6)}{=}
= \M e^{it\xi_1} \cdot \M e^{it\xi_2} = \Theta_{\xi_1} (t) \cdot \Theta_{\xi_2} (t).$$
\end{proof}

\begin{consq}
\label{consq:24.8}
Если $\xi_1 , \ldots , \xi_n$ — независимые случайные величины, то
$$\Theta_{\xi_1 + \ldots +\xi_2} (t) = \Theta_{\xi_1} (t) \cdot \ldots \cdot \Theta_{\xi_n} (t).$$
\end{consq}

\begin{lemma}
\label{lemma:24.9}
$\Theta_{\xi} (−t) = \overline{\Theta_{\xi}(t)}$, где черта означает комплексное сопряжение.
\end{lemma}

\begin{proof}
$\Theta_{\xi} (−t) = \M e^{−it\xi} = \M \overline{e^{it\xi}} = \overline{\M e^{it\xi}} = \overline{\Theta_{\xi} (t)}.$
\end{proof}

\begin{lemma}
\label{lemma:24.10}
Если $\xi$ — дискретная случайная величина, заданная рядом распределения (конечным или бесконечным)

\begin{center}
	\begin{tabular}{|c|c|c|c|c|}
		\hline
		$\xi$ & $x_1$ & $x_2$ & $\ldots$ & $x_k$ \\ \hline
		$P$  & $p_1$ & $p_2$  & $\ldots$ & $p_k$ \\ \hline
	\end{tabular}
\end{center}

то $\Theta_{\xi} (t) = \sum_k e^{itx_{k}} p_k$
\end{lemma}

\begin{proof}
Это утверждение непосредственно следует из теор. \ref{th:17.4}.1).
\end{proof}

\begin{lemma}
\label{lemma:24.11}
Пусть существует начальный момент $n$-го порядка,
$n = 1, 2, \ldots$, случайной величины $\xi$, т.е. $\M|\xi|^n < \infty$. Тогда её характеристическая функция $\Theta_{\xi} (t)$ является $n$ раз непрерывно дифференцируемой, и
её $n$-я производная в нуле равна
$$
\Theta_{\xi}^{(n)} (0) = i^n \M \xi^n $$
\end{lemma}

\begin{proof}
Заметим сначала, что т.к. существует начальный момент
$M\xi^n$ , то существуют все моменты $M\xi^k$ при $k < n$.

По лемме \ref{lemma:24.5}.2) имеем $|\Theta_{\xi} (t)| \leq 1$, поэтому интеграл в правой части 

$$\Theta_{\xi} (t) = \int\limits_{-\infty}^{\infty} e^{itx}f_{\xi}(x)dx$$

равномерно сходится по параметру $t$ и следовательно его можно дифференцировать по $t$:
\begin{gather*}
	\Theta_{\xi}' (t) = i \int\limits_{-\infty}^{\infty} xe^{itx} f_{\xi}(x)dx,\\
	\Theta_{\xi}'' (t) = i^2 \int\limits_{-\infty}^{\infty} x^2 e^{itx} f_{\xi}(x)dx,\\
	\ldots\ldots\ldots\ldots\ldots\ldots\ldots\\
	\Theta_{\xi}^(n) (t) = i^n \int\limits_{-\infty}^{\infty} x^n e^{itx} f_{\xi}(x)dx.
\end{gather*}

Подставляя в эти равенства $t = 0$, получим требуемый результат: 
$\Theta_{\xi}' (0) = i\M \xi, \Theta_{\xi}'' (0) = i^2 \M \xi^2 , \ldots , \Theta_{\xi}^{(n)} (0) = i^n \M \xi^n .$
\end{proof}

\begin{lemma}
\label{lemma:24.12}
Если существуют начальные моменты $\M \xi, \M \xi^2 , \ldots , \M \xi^n$ случайной величины $\xi$, т.е. $M |\xi|^n < \infty$, тогда в окрестности нуля её
характеристическая функция $\Theta_{\xi} (t)$ разлагается в ряд Тейлора\footnote{
Брук Тэйлор (Brook Taylor, 1685 — 1731), английский математик, именем которого назван ряд (опубликованный им в 1715—1717 гг.), однако этот ряд был известен и применялся ещё в XVII веке Грегори и Ньютоном.
}
$$\Theta_{\xi} (t) = 1 + \frac{i\M \xi}{1!}t + \frac{i^2 \M \xi^2}{2!}t^2 + \ldots + \frac{i^n \M \xi^n}{n!}t^n + \text{o}(t^n)$$
\end{lemma}

\begin{proof}
В стандартное разложение функции $\Theta_{\xi} (t)$ в ряд Тейлора
$$\Theta_{\xi} (t) = 1 + \frac{\Theta_{\xi}'(0)}{1!}t + \frac{\Theta_{\xi}''(0)}{2!}t^2 + \ldots + \frac{\Theta_{\xi}(n)(0)}{n!}t^n + \text{o}(t^n)$$ 

подставим выражения для $\Theta_{\xi}' (0), \Theta_{\xi}'' (0), \ldots , \Theta_{\xi} (0)$, доставляемые леммой \ref{lemma:24.11}, получим требуемый результат.
\end{proof}

% \textit{
В заключение параграфа сформулируем без доказательства теорему Леви, которая устанавливает <<непрерывное>> соответствие между слабо сходящимися последовательностями случайных величин и поточечно сходящимися последовательностями характеристических функций. <<Непрерывность>> этого соответствия состоит в том, что пределу при слабой сходимости соответствует предел при поточечной сходимости и наоборот.
% }

\begin{theorem}[Леви\footnote{Поль Пьер Леви (Paul Pierre Levy, 1886 — 1971), выдающийся французский математик, основные
труды по теории вероятностей, функциональному анализу, теории функций и механике.
} о непрерывном соответствии]
\label{th:24.13}
Последовательность случайных величин $\{\xi_n \}_{n=1}^{\infty}$ слабо сходится к случайной величине $\xi$
тогда и только тогда, когда последовательность их характеристических функций $\{\Theta_{\xi_n} (t)\}_{n=1}^{\infty}$ поточечно сходится к характеристической функции $\Theta_{\xi} (t)$.
\end{theorem}