%!TEX root = ../var.tex

\begin{definition} 
\label{def:5.1}
События A и B называются \textit{независимыми}, если выполняется тождество
\begin{equation*}
	P(A \cap B) = P(A)P(B),
\end{equation*}
в противном случае они называются \textit{зависимыми}.
\end{definition}
	
	\textbf{Примеры}. 

	1) Рассмотрим колоду карт в 52 листа. Пусть событие A означает вытянуть пику ♠, а B означает вытянуть даму \textbf{Д}, Тогда событие $A\cap B$
означает вытянуть пиковую даму \textbf{Д♠}. Легко подсчитать вероятности этих
событий $P(A\cap B)$ = 1/52; P(A) = 1/4 и P(B) = 1/13. Подставив эти веро-
ятности в равенство опред. \ref{def:5.1}, получим тождество $1/52 = 1/4 \cdot1/13$; т.е. по
опред. \ref{def:5.1} события A и B независимы (в этой колоде).


2) Рассмотрим полную колоду карт в 54 листа (с двумя шутами). Пусть
события A и B — те же как в предыдущем пункте. Легко подсчитать, что в
этой колоде $P(A \cap B) = 1/54$, $P(A) = 13/54$ и $P(B) = 2/27$. Таким образом
$1/54 \neq 13/54 \cdot 2/27$, и по опред. \ref{def:5.1} события A и B являются зависимыми в
полной колоде.
Получилось, что свойство быть или не быть независимыми зависит не от
самих событий, а от строения пространства $\Omega$ (52 или 54).

\begin{lemma}
Если события A и B независимы, то пары событий A и B,
A и B, A и B тоже являются независимыми.
\end{lemma}
\begin{proof}
 Докажем, что события A и $\overline{B}$ независимы. Так как 
 $A =(A \cup B)\cup(A \cap \overline{B})$, и события $A\cup B$ и $A\cup \overline{B}$ несовместны, то  $P(A) = P(A\cap B)+P(A\cap\overline{B})$. 
Поэтому $P(A \cap  \overline{B}) = P(A) − P(A \cap  B) = P(A) − P(A)P(B) =
P(A)(1 − P(B)) = P(A)P(\overline{B})$.
Независимость пар событий $\overline{A}$ и B, A и $\overline{B}$ доказывается аналогично. Доказать самостоятельно.
\end{proof}

\begin{definition}
События $A_1,\dots,A_n$ называются независимыми в совокупности, если для $1 \leqslant i_1 < \dots < i_k \leqslant n$ выполнено равенство

$P(A_{i1}\cap \dots \cap  A_{ik}) = P(A_{i1}) \cdot\dots\cdot P(A_{ik})$.
\end{definition}
\begin{remark} 
Если события $A_1, \dots ,A_n$ независимыми в совокупности,
то они попарно независимы. Чтобы это увидеть, достаточно в последнем равенстве положить $k = 2$. Обратное, как показывает следующий пример, не
верно.
\end{remark}

\begin{figure}[H]
	\centering
	\includegraphics[width=0.5\textwidth]{pic/pic6.pdf}
	\caption{Тетраэдр Бернштейна}
	\label{pic:6}
\end{figure}
\begin{theorem}[Тетраэдр Бернштейна
\footnote{Сергей Натанович Берншейн (1880 — 1968), советский математик.}]
\end{theorem}
	Рассмотрим правильный тетраэдр, три
грани которого окрашены в эти синий, зелёный, и красный цвет, см. рис.\ref{pic:6},
а четвёртая грань разделена на три треугольника, и эти треугольники окрашены в те же три цвета. Обозначим через B, G и R события означающие
выпадение снизу грани, содержащей соответственно синий (blue), зелёный
(green), и красный (red) цвета.
Легко видеть, что каждый цвет нарисован на двух из четырёх граней
поэтому P(B) = P(G) = P(R) = 1/2. Также легко видеть, что появление
любой пары из них имеет вероятности $P(B\cap G) = P(G\cap R) = P(B\cap R) = 1/4$.
Т.о., для каждой пары из этих событий формула опред. 5.1 выполнена, и
следовательно события попарно независимы.

Теперь проверим независимость событий B, G и R в совокупности. Легко
видеть, что вероятность $P(B \cap  G \cap  R)$ выпадения трёхцветной грани равна
1/4. Это левая часть равенства опред. 5.3. Правая часть равна $P(B) \cdot P(G) \cdot
P(R) = 1/8$. Т.е. равенство опред. 5.2 не выполнено, значит события B, G
и R зависимы в совокупности.



