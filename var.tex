% Тип документа
\documentclass[a4paper,14pt]{extarticle}

% Шрифты, кодировки, символьные таблицы, переносы
\usepackage{cmap}
\usepackage[T2A]{fontenc}
\usepackage[utf8x]{inputenc}
\usepackage[russian]{babel}

% Это пакет -- хитрый пакет, он нужен но не нужен
\usepackage[mode=buildnew]{standalone}

\usepackage
	{
		% Дополнения Американского математического общества (AMS)
		amssymb,
		amsfonts,
		amsmath,
		% amsthm,
		% misccorr,
		% 
		% Графики и рисунки
		wrapfig,
		graphicx,
		subcaption,
		float,
		tikz,
		tikz-3dplot,
		caption,
		csvsimple,
		color,
		booktabs,
		pgfplots,
		pgfplotstable,
		geometry,
		% 
		% Таблицы, списки
		makecell,
		multirow,
		indentfirst,
		%
		% Интегралы и прочие обозначения
		ulem,
		esint,
		esdiff,
		% 
		% Колонтитулы
		fancyhdr,
	}  


% Обводка текста в TikZ
\usepackage[outline]{contour}

% Увеличенный межстрочный интервал, французские пробелы
% \linespread{1.3} 
\frenchspacing 

 
\usetikzlibrary
	{
		decorations.pathreplacing,
		decorations.pathmorphing,
		patterns,
		calc,
		scopes,
		arrows,
		fadings,
		through,
		shapes.misc,
		arrows.meta,
		3d,
		quotes,
		angles,
		babel
	}


\tikzset{
	force/.style=	{
		>=latex,
		draw=blue,
		fill=blue,
				 	}, 
	%				 	
	axis/.style=	{
		densely dashed,
		blue,
		line width=1pt,
		font=\small,
					},
	%
	th/.style=	{
		line width=1pt},
	%
	acceleration/.style={
		>=open triangle 60,
		draw=magenta,
		fill=magenta,
					},
	%
	inforce/.style=	{
		force,
		double equal sign distance=2pt,
					},
	%
	interface/.style={
		pattern = north east lines, 
		draw    = none, 
		pattern color=gray!60,
					},
	cross/.style=	{
		cross out, 
		draw=black, 
		minimum size=2*(#1-\pgflinewidth), 
		inner sep=0pt, outer sep=0pt,
					},
	%
	cargo/.style=	{
		rectangle, 
		fill=black!70, 
		inner sep=2.5mm,
					},
	%
	caption/.style= {
		midway,
		fill=white!20, 
		opacity=0.9
					},
	%
	}

\newenvironment{tikzpict}
    {
	    \begin{figure}[htbp]
		\centering
		\begin{tikzpicture}
    }
    { 
		\end{tikzpicture}
		% \caption{caption}
		% \label{fig:label}
		\end{figure}
    }


\newcommand{\vbLabel}[3]{\draw ($(#1,#2)+(0,5pt)$) -- ($(#1,#2)-(0,5pt)$) node[below]{#3}}
\newcommand{\vaLabel}[3]{\draw ($(#1,#2)+(0,5pt)$) node[above]{#3} -- ($(#1,#2)-(0,5pt)$) }

\newcommand{\hrLabel}[3]{\draw ($(#1,#2)+(5pt,0)$) -- ($(#1,#2)-(5pt,0)$) node[right, xshift=1em]{#3}}
\newcommand{\hlLabel}[3]{\draw ($(#1,#2)+(5pt,0)$) node[left, xshift=-1em]{#3} -- ($(#1,#2)-(5pt,0)$) }



\newcommand\zi{^{\,*}_i}
\newcommand\sumn{\sum_{i=1}^{N}}

\tikzset{
	coordsys/.style={scale=1.8,x={(1.1cm,-0cm)},y={(0.5cm,1cm)}, z={(0cm,0.8cm)}},
	coordsys/.style={scale=1.5,x={(0cm,0cm)},y={(1cm,0cm)}, z={(0cm,1cm)}}, 
	coordsys/.style={scale=1.5,x={(1cm,0cm)},y={(0cm,1cm)}, z={(0cm,0cm)}}, 
}

\usepgfplotslibrary{units}


% Draw line annotation
% Input:
%   #1 Line offset (optional)
%   #2 Line angle
%   #3 Line length
%   #5 Line label
% Example:
%   \lineann[1]{30}{2}{$L_1$}

\newcommand{\lineann}[4][0.5]{%
    \begin{scope}[rotate=#2, blue,inner sep=2pt, ]
        \draw[dashed, blue!40] (0,0) -- +(0,#1)
            node [coordinate, near end] (a) {};
        \draw[dashed, blue!40] (#3,0) -- +(0,#1)
            node [coordinate, near end] (b) {};
        \draw[|<->|] (a) -- node[fill=white, scale=0.8] {#4} (b);
    \end{scope}
}

\newcommand{\lineannn}[4][0.5]{%
    \begin{scope}[rotate=#2, blue,inner sep=2pt, ]
        \draw[dashed, blue!40] (0,0) -- +(0,#1)
            node [coordinate, near end] (a) {};
        \draw[dashed, blue!40] (#3,0) -- +(0,#1)
            node [coordinate, near end] (b) {};
        % \draw[color=white, color=blue] (a) -- node[fill=white, scale=0.8] {#4} (b);
        \draw[->|] (a)++(-0.3,0) -- (a);
        \draw[->|] (b)++(0.3,0) coordinate (xx) -- (b);
        \draw (xx) node[fill=white, scale=0.8, right] {#4};
    \end{scope}
}

% Круговая стрелка относительно центра (дуга из центра)
\tikzset{
  pics/carc/.style args={#1:#2:#3}{
    code={
      \draw[pic actions] (#1:#3) arc(#1:#2:#3);
    }
  },
  dash/.style={
  	dash pattern=on 5mm off 5mm
  }
}

% Среднее <#1>
\newcommand{\mean}[1]{\langle#1\rangle}

\pgfplotsset{
    % most recent feature set of pgfplots
    compat=newest,
}

% const прямым шрифтом
\newcommand\ct[1]{\text{\rmfamily\upshape #1}}
\newcommand*{\const}{\ct{const}}


\usepackage[europeanresistors,americaninductors]{circuitikz}

% Style to select only points from #1 to #2 (inclusive)
\pgfplotsset{select/.style 2 args={
    x filter/.code={
        \ifnum\coordindex<#1\def\pgfmathresult{}\fi
        \ifnum\coordindex>#2\def\pgfmathresult{}\fi
    }
}}


\usepackage{array}



%%%%%%%%%%%%%%%%%%%%%%%%%%%%%%%%%%%%%%%%%%%%%%%%%
\makeatletter
\newif\if@gather@prefix 
\preto\place@tag@gather{% 
  \if@gather@prefix\iftagsleft@ 
    \kern-\gdisplaywidth@ 
    \rlap{\gather@prefix}% 
    \kern\gdisplaywidth@ 
  \fi\fi 
} 
\appto\place@tag@gather{% 
  \if@gather@prefix\iftagsleft@\else 
    \kern-\displaywidth 
    \rlap{\gather@prefix}% 
    \kern\displaywidth 
  \fi\fi 
  \global\@gather@prefixfalse 
} 
\preto\place@tag{% 
  \if@gather@prefix\iftagsleft@ 
    \kern-\gdisplaywidth@ 
    \rlap{\gather@prefix}% 
    \kern\displaywidth@ 
  \fi\fi 
} 
\appto\place@tag{% 
  \if@gather@prefix\iftagsleft@\else 
    \kern-\displaywidth 
    \rlap{\gather@prefix}% 
    \kern\displaywidth 
  \fi\fi 
  \global\@gather@prefixfalse 
} 
\newcommand*{\beforetext}[1]{% 
  \ifmeasuring@\else
  \gdef\gather@prefix{#1}% 
  \global\@gather@prefixtrue 
  \fi
} 
\makeatother
%%%%%%%%%%%%%%%%%%%%%%%%%%%%%%%%%%%%%%%%%%%%%%%%%

\geometry		
	{
		left			=	2cm,
		right 			=	2cm,
		top 			=	3cm,
		bottom 			=	3cm,
		bindingoffset	=	0cm
	}

%%%%%%%%%%%%%%%%%%%%%%%%%%%%%%%%%%%%%%%%%%%%%%%%%%%%%%%%%%%%%%%%%%%%%%%%%%%%%%%



	%применим колонтитул к стилю страницы

%%%%%%%%%%%%%%%%%%%%%%%%%%%%%%%%%%%%%%%%%%%%%%%%%%%%%%%%%%%%%%%%%%%%%%%%%%%%%%%

\renewcommand{\contentsname}{Оглавление}

\usepackage{tocloft}
% \renewcommand{\cftpartleader}{\cftdotfill{\cftdotsep}} % for parts
% \renewcommand{\cftsectiondotsep}{\cftdotsep}% Chapters should use dots in ToC
\renewcommand{\cftsecleader}{\cftdotfill{\cftdotsep}}
%\renewcommand{\cftsecleader}{\cftdotfill{\cftdotsep}} % for sections, if you really want! (It is default in report and book class (So you may not need it).
% ---------
% \newcommand{\cftchapaftersnum}{.}%
% \usepackage{titlesec}
% \titlelabel{\thetitle.\quad}
\usepackage{secdot}
\sectiondot{subsection}
\usepackage{setspace}
\usepackage{amsmath}

\DeclareMathOperator{\sinc}{sinc}
\newcommand{\dif}[3]{


\pgfplotstablegetelem{0}{#2}\of#1

% add column LocalDistance
\pgfplotstablecreatecol
    [expr={\thisrow{#2} - \pgfplotsretval}]
    {LocalDistance#3}{#1}

% add column DifferenceDistance
\pgfplotstablecreatecol
    % [expr={-\thisrow{LocalDistance} + \prevrow{LocalDistance}}]
    % [expr={rad(180)}]
    [expr={-\thisrow{LocalDistance#3} + \prevrow{LocalDistance#3}}]
    {#3}{#1}

}
\newcommand{\Exp}[1]{
	\exp\left(#1\right)
}
\newcommand{\Sinc}[1]{
	\sinc\left(#1\right)
}
\newcommand{\Sin}[1]{
	\sin\left(#1\right)
}
\begin{document}

\newtheorem{theorem}{Теорема}[section]

\newtheorem{lemma}[theorem]{Лемма}%[section]

\theoremstyle{definition}
\newtheorem{consq}[theorem]{Следствие}%[section]

\theoremstyle{remark}
\newtheorem{remark}[theorem]{Замечание}%[section]

\def\labauthors{Понур К.А., Сарафанов Ф.Г., Сидоров Д.А.}
\def\labgroup{420}
\def\labnumber{320}
\def\labtheme{Дифракций Фраунгофера}
\renewcommand{\vec}{\mathbf}
\renewcommand{\Re}{\operatorname{Re}}
\renewcommand{\Im}{\operatorname{Im}}
\renewcommand{\phi}{\varphi}
\renewcommand{\kappa}{\varkappa}
\renewcommand{\hat}{\widehat}
%%%%%%%%%%%%%%%%%%%%%%%%%%%%%%%%%%%%%%%%%%%%%%%%%%%%%%%%%%%%%%%%%%%%%%%%%%%%%%%
%!TEX root = ../var.tex
\begin{titlepage}

\begin{center}

{\small\textsc{Нижегородский государственный университет имени Н.\,И. Лобачевского}}
\vskip 1pt \hrule \vskip 3pt
{\small\textsc{Радиофизический факультет}}

\vfill

{ \LargeМетодическое пособие \vskip 12pt\bfseries \Huge Теория вероятностей}
	
\end{center}

\vfill
	
% \begin{flushright}
% 	{Выполнили студенты \labgroup\ группы\\ \labauthors}%\vskip 12pt Принял:\\ Менсов С.\,Н.}
% \end{flushright}
	
\vfill
	
\begin{center}
	Нижний Новгород, \the\year
\end{center}

\end{titlepage}


%%%%%%%%%%%%%%%%%%%%%%%%%%%%%%%%%%%%%%%%%%%%%%%%%%%%%%%%%%%%%%%%%%%%%%%%%%%%%%%
\begin{spacing}{1}
\tableofcontents
\end{spacing}
% \setstretch{1.2}
\newpage

\part*{Исторические сведения}
Возникновение теории вероятностей как науки относят к средним векам,
когда появилась возможность и возникла необходимость изучения математи-
ческими методами азартных игр (таких как орлянка, кости, рулетка). Самые
ранние работы учёных в области теории вероятностей относятся к XVII ве-
ку. Первоначально её основные понятия не имели строго математического
описания. Задачи, из которых позже выросла теория вероятностей представ-
ляли набор некоторых эмпирических фактов о свойствах реальных событий,
которые формулировались с помощью наглядных описаний. Исследуя про-
гнозирование выигрыша при бросании костей в письмах друг другу, Блез
Паскаль и Пьер Ферма открыли первые вероятностные закономерности. Ре-
шением тех же задач занимался и Христиан Гюйгенс. При этом с перепиской
Паскаля и Ферма он знаком не был и методику решения изобрёл самосто-
ятельно. Его статья, в которой он ввёл основные понятия теории вероятно-
стей (понятие вероятности как величину шанса; математическое ожидание
для дискретных случаев в виде цены шанса). В своей статье он использует
(не сформулированные ещё в явном виде) теоремы сложения и умножения
вероятностей. Статья была опубликована в печатном виде на двадцать лет
раньше (1657 г.) издания писем Паскаля и Ферма (1679 г.).
Важный вклад в теорию вероятностей внёс Якоб Бернулли, он дал до-
казательство закона больших чисел в простейшем случае независимых ис-
пытаний. В первой половине XIX века теория вероятностей начинает приме-
няться к анализу ошибок наблюдений; Лаплас и Пуассон доказали первые
предельные теоремы. Во второй половине XIX века основной вклад внесли
русские учёные П. Л. Чебышёв, А. А. Марков и А. М. Ляпунов. В это вре-
мя были доказаны закон больших чисел, центральная предельная теорема,
а также разработана теория цепей Маркова. Современный вид теория веро-
ятностей получила благодаря аксиоматике, предложенной Андреем Никола-
евичем Колмогоровым. В результате теория вероятностей приобрела строгий
математический вид и окончательно стала восприниматься как один из раз-
делов математики.

Википедия,
Статья "Теория вероятностей".

\part{События и их вероятности}
\section{Элементы комбинаторики. Схемы шансов}

В этом параграфе мы подсчитываем число элементарных событий или,
проще говоря, исходов, шансев, которые могут возникать в результате эксперимента. Например, при подбрасывании монеты могут произойти 2 исхода,
при подбрасывании игрального кубика могут произойти 6 исходов, при извлечении карты из колоды в 54 листа могут произойти 53 исхода. Такие подсчёты изучают в разделе математики, называемом комбинаторикой.
Пусть $A$ и $B$ — два непересекающихся конечных множества с числом
элементов $m$ и $n$ соответственно. Очевидны следующие две леммы.



\begin{lemma}(о сумме). Число шансов выбрать один элемент либо из $A$
либо из $B$, т.е. из объединения $A\cup B$, равно $m+n$.
\end{lemma}
\begin{lemma}(о произведении). Число шансов выбрать пару элементов,
один из $A$, а другой из $B$, равно $mn$, т.е. числу элементов в декартовом
произведении $A\times B$.
\end{lemma}
Непосредственным обобщением предыдущей леммы является следующая
теорема.
\begin{theorem}
\label{t:1}
Пусть $A_2,A_2,\dots,A_k$ — конечные непересекающиеся множества, имеющие $n_1
,n_2, \dots, n_k$ элементов соответственно. Выберем из
каждого множества по одному элементу. Тогда общее число способов, которыми можно осуществить такой выбор, равно $n_1n_2\dots n_k$.
\end{theorem}

\begin{proof}

Ясно, что число способов такого выбора равно числу точек (элементов) в декартовом произведении $A_1\times A_2\times\dots A_k$, т.е. равно $
n_1\cdot n_2\dots n_k$.
\end{proof}
\subsection{Эксперименты выбора шариков}
Рассмотрим ящик, содержащий $n$ одинаковых шариков, на которых написаны
числа $1, 2,\dots, n$. Эксперимент состоит в том, что из ящика, не глядя, по
одному вынимают $k$ шариков, где $k\leqslant n$. Обозначим через
\begin{gather*}
(n_1, n_2,\dots, n_k)
\end{gather*}
упорядоченный набор чисел, где $n_1$ — номер 1-го вынутого шарика, $n2$ —
номер 2-го шарика,$\dots$, $n_k$ — номер $k$-го шарика.
Например, из 5 занумерованных шариков выбрали 3 шарика и получился
набор (4, 2, 1).

Сколько имеется различных способов вынуть из ящика $k$ шариков? На
этот вопрос нельзя дать однозначный ответ, потому что такой эксперимент
определён неоднозначно.

Во-первых, не определено, возвращают ли извлеченный шарик обратно в
ящик. Во-вторых, не определено, какие наборы номеров считать различными
и какие наборы считать одинаковыми.

Рассмотрим следующие возможные условия проведения эксперимента.
\begin{enumerate}
\item 
\textit{Эксперимент с возвращением}. 
Каждый извлечённый шарик возвращается в ящик.
В этом случае в наборе могут появляться одинаковые номера. Например, при выборе трёх шариков из ящика, содержащего пять шариков
с номерами 1, 2, 3, 4 и 5, могут появиться наборы (3, 3, 5), (1, 2, 4)
и (4, 2, 1).
\item 
\textit{Эксперимент без возвращений. Извлечённые шарики в ящик не воз-
вращаются}.
В этом случае в наборе не могут встречаться одинаковые номера. В
рассмотренном выше примере набор (3,3,5) не может появиться, а
наборы (1,2,4) и (4,2,1) могут.

\end{enumerate}

Опишем теперь, какие наборы номеров мы будем считать различными.
Существуют ровно две возможности.
\begin{enumerate}
\item \textit{Эксперимент с учётом порядка}. Два набора номеров считаются различными, если они отличаются либо составом, либо порядком.
В рассмотренном выше примере все наборы (3,3,5), (1,2,4) и (4,2,1)
считаются различными.
\item
\textit{Эксперимент без учёта порядка}. Два набора номеров считаются различными, если они отличаются только составом.

В рассмотренном выше примере наборы (1,2,4) и (4,2,1) доставляют
одно и тот же элементарное событие, а набор (3,3,5) — другое.
\end{enumerate}

Подсчитаем теперь, сколько получится различных исходов для каждого
из четырёх экспериментов. Заметим, что в литературе такие эксперименты
часто называют схемами выбора или схемами шансов. Схема шансов — это
условия (с возвратом или без, какие наборы различны и т.д.), при которых
проводится эксперимент.

\subsection{Схема шансов без возвращения и с учетом порядка}
\begin{theorem}
В эксперименте без возвращения и с учётом порядка число способов выбрать $k$ элементов из $n$-элементного множества равно
\begin{equation}
	A_n^k=n(n-1)\dots(n-k+1)=\frac{n!}{(n-k)!}
\end{equation}
\end{theorem}
Число $A_n^k$ называется \textit{числом размещений элементов $k$ на $n$ местах}. Читается: <<$A$ из $n$ по $k$>>.  

\begin{proof}
При выборе первого шарика имеется $n$ возможностей При выборе первого шарика имеется n возможностей. При выборе второго шарика остаётся $n−1$ возможностей, и т.д. При выборе последнего $k$-го шарика остаётся $n − k + 1$ возможностей. По теор. \ref{t:1} общее
число наборов равно $n(n−1)\dots(n−k +1)$, что и требовалось доказать.
\end{proof}

\begin{consq}
\label{cosq:1}
Число перестановок из $n$ элементов равно $n!$.
\end{consq}

\begin{proof}
Очевидно, что перестановка есть результат выбора по схеме без возвращения и с учётом порядка всех $n$ элементов из $n$, т.е. общее
число перестановок равно $A_n^2=n!$.
\end{proof}
\subsection{Схема шансов без возвращения и без учёта порядка}
\begin{theorem}
\label{t:2}
В эксперименте без возвращения и без учёта порядка число
способов извлечь k из n-элементного множества равно
\begin{equation}
	C_n^k=\frac{A_n^k}{k!}=\frac{n!}{k!(n-k)!}
\end{equation}
	
Число $C_n^k$ называется \textit{числом сочетаний k элементов из n элементов.}
Читается: <<$C$ из $n$ по $k$>>
\end{theorem}

\begin{proof}
По следствию \ref{cosq:1} из $k$ элементов можно образовать $k!$ упорядоченных наборов. Поэтому количество сочетаний 
(неупорядоченных наборов)
в $k!$ раз меньше, чем число размещений. Поделив $A^k_
n$ на $k!$, получим требуемый результат.
\end{proof}

\subsection{Схема шансов с возвращением и с учётом порядка}
\begin{theorem}
В эксперименте с возвращением и с учётом порядка число
способов извлечь k элементов из n-элементного множества равно $n^k$.
\end{theorem}
\begin{proof}
При выборе каждого из $k$ шариков имеется $n$ возможностей. 
По теореме \ref{t:1} общее число наборов равно $n\cdot n\cdot n\dots\cdot n=n^k$.
\end{proof}
\subsection{Схема шансов с возвращением и без учёта порядка}
\begin{remark}
\label{remark:1}
Рассмотрим для примера ящик с двумя шариками 1 и 2,
из которого мы вынимаем последовательно два шарика. Без учёта порядка
имеется 3 исхода: 
\\* $\{1,1\}, \{1,2\} = \{2,1\}, \{2, 2\}$.
\end{remark}

\begin{theorem}
В эксперименте с возвращением и без учёта порядка число
способов извлечь $k$ элементов из $n$-элементного множества равно $C^k_{n+k-1}$.
\end{theorem}
\begin{proof}
Т.к. порядок появления шариков не учитывается, то мы
учитываем лишь только то, сколько раз в наборе появится $i$-й шарик для
каждого $i = 1, 2,\dots, n$. Обозначим через $k_i$ число появлений $i$-го шарика в
наборе. Во-первых, $0 \leqslant k_i \leqslant k$, а во-вторых,
\begin{gather*}
k_1+k_2+\dots+k_n=k.
\end{gather*}

Поставим каждому исходу в соответствие набор чисел $(k_11, k_2,\dots, k_n)$.
Легко видеть, что это соответствие является взаимно однозначным. Такое соответствие можно рассматривать как способ нумерации наборов. (Например,
исходам из замеч. \ref{remark:1} ставятся в соответствие следующие номера:
$\{1, 1\} \leftrightarrow (2, 0), 
\{1, 2\} \leftrightarrow (1, 1) и 
\{2, 2\} \leftrightarrow (0, 2).$
Рассмотрим теперь другой эксперимент. Пусть теперь имеется $n$ урн с
номерами $i = 1, 2,\dots, n$, в которых размещаются $k$ неразличимых шариков.
Сколько существует способов разложить шарики по урнам? Нас интересует
только количество шариков в $i$-й урне для каждого $i$. Обозначим через $k_i$
число шариков в $i$-й урне. Ясно, что $0 \leqslant ki \leqslant k$, и что числа $k_1$ и в этом
эксперименте тоже удовлетворяют уравнению
\begin{gather*}
k_1+k_2+\dots+k_n=k.
\end{gather*}

Исходы этого эксперимента тоже взаимно однозначно описываются наборами
чисел \newline $(k_1, k_2,\dots, k_n)$. Т.о., исходы в эксперименте с урнами и исходы предыдущего эксперимента с ящиком занумерованы одним и тем же набором чисел,
поэтому число исходов в обоих экспериментах одно и то же и равно числу
решений этого уравнения. Вычислим это число для эксперимента с урнами.
Изобразим расположение шариков в урнах с помощью схематичного рисунка. Вертикальными линиями обозначим перегородки между урнами, а
кружками — шарики, находящиеся в них. Например,
\begin{gather*}
\left|\bullet\bullet\right|\bullet\bullet\bullet||\bullet||\bullet\bullet|\bullet|.
\end{gather*}

На рисунке показаны 9 шариков, рассыпанные по 7 урнам: 1-я и 6-я урны
содержат по 2 шарика, 2-я урна содержит 3 шарика, 3-я и 5-я урны — пустые
и, наконец, 4-я и 7-я урны содержат по одному шарику.

Меняя местами шарики и стенки, можно получить все возможные расположения шариков в урнах. Другими словами, все расположения можно
получить, расставляя $k$ шариков и $n − 1$ стенок на $n − 1 + k$ местах. Число
$n−1+k$ получается следующим образом. Число стенок у $n$ урн равно $n+1$,
и т.к. две крайние стенки двигать нельзя, то число стенок, которые можно
двигать равно $n − 1$. Поэтому шарики могут занимать $k$ мест, а стенки урн
— оставшиеся $n−1$ место. По теор. \ref{t:2} число способов расставить $k$ шариков
на $n − 1 + k$ местах и затем расставить стенки на оставшихся $n − 1$ местах
равно $C^k_{n+k-1}$. Что и требовалось доказать.
\end{proof}
\subsection{Элементы комбинаторики. Схемы шансов}


\section{События, операции над ними и $\sigma$-алгебры событий}

\section{Вероятность и её свойства}


\section{Способы задания и подсчёта вероятности}

\subsection{Экспериментальное нахождение вероятности}


\subsection{Вероятность на конечном пространстве}

\subsection{Классическая вероятность}

\subsection{Вероятность на счётном пространстве}

\subsection{Геометрическая вероятность}

\section{Независимые события}

\section{Условная вероятность}

\section{Формула полной вероятности и формулы Байеса}

\section{Биномиальное распределение}

\section{$k$-номинальное распределение}

\section{Гипергеометрическое распрделение}

\part{Теория случайных величин}

\section{Случайные величины}

\section{Абсолютно непрерывные случайные величины}

\section{Функции Хевисайда и Дирака}

\section{Функции одной случайной величины}

\section{Случайные векторы и их распределения}

\section{Функции от двух случайных величин}

\section{Математическое ожидание}

\section{Дисперсия}

\section{Числовые характеристики зависимости
случайных величин}


\part{Законы больших чисел}

\section{Неравенство Бьенеме–Чебышёва и
неравенство Маркова}

\section{Последовательности случайных величин}

\section{Законы больших чисел}

\section{Предельные теоремы для
биномиального распределения}

\section{Характеристические функции}

\section{Вычисление характеристических
функций}

\section{Центральная предельная теорема}

\section{Сферическое, $\xi^2$-распределение
и распределение Стьюдента}

\section{Цепи Маркова}









\end{document}