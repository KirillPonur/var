\input{text/diss}
\usepackage{setspace}
\usepackage{amsmath}

\DeclareMathOperator{\sinc}{sinc}
\newcommand{\dif}[3]{


\pgfplotstablegetelem{0}{#2}\of#1

% add column LocalDistance
\pgfplotstablecreatecol
    [expr={\thisrow{#2} - \pgfplotsretval}]
    {LocalDistance#3}{#1}

% add column DifferenceDistance
\pgfplotstablecreatecol
    % [expr={-\thisrow{LocalDistance} + \prevrow{LocalDistance}}]
    % [expr={rad(180)}]
    [expr={-\thisrow{LocalDistance#3} + \prevrow{LocalDistance#3}}]
    {#3}{#1}

}
\newcommand{\Exp}[1]{
	\exp\left(#1\right)
}
\newcommand{\Sinc}[1]{
	\sinc\left(#1\right)
}
\newcommand{\Sin}[1]{
	\sin\left(#1\right)
}
\begin{document}

\newtheorem{theorem}{Теорема}[section]
\newtheorem{lemma}{Лемма}[section]

\def\labauthors{Понур К.А., Сарафанов Ф.Г., Сидоров Д.А.}
\def\labgroup{420}
\def\labnumber{320}
\def\labtheme{Дифракций Фраунгофера}
\renewcommand{\vec}{\mathbf}
\renewcommand{\Re}{\operatorname{Re}}
\renewcommand{\Im}{\operatorname{Im}}
\renewcommand{\phi}{\varphi}
\renewcommand{\kappa}{\varkappa}
\renewcommand{\hat}{\widehat}
%%%%%%%%%%%%%%%%%%%%%%%%%%%%%%%%%%%%%%%%%%%%%%%%%%%%%%%%%%%%%%%%%%%%%%%%%%%%%%%
%!TEX root = ../var.tex
\begin{titlepage}

\begin{center}

{\small\textsc{Нижегородский государственный университет имени Н.\,И. Лобачевского}}
\vskip 1pt \hrule \vskip 3pt
{\small\textsc{Радиофизический факультет}}

\vfill

{ \LargeМетодическое пособие \vskip 12pt\bfseries \Huge Теория вероятностей}
	
\end{center}

\vfill
	
% \begin{flushright}
% 	{Выполнили студенты \labgroup\ группы\\ \labauthors}%\vskip 12pt Принял:\\ Менсов С.\,Н.}
% \end{flushright}
	
\vfill
	
\begin{center}
	Нижний Новгород, \the\year
\end{center}

\end{titlepage}


%%%%%%%%%%%%%%%%%%%%%%%%%%%%%%%%%%%%%%%%%%%%%%%%%%%%%%%%%%%%%%%%%%%%%%%%%%%%%%%
\begin{spacing}{1}
\tableofcontents
\end{spacing}
% \setstretch{1.2}
\newpage
\begin{theorem} \label{t1} % сразу указываем ссылку
Пусть у нас есть два множества, построим..... (текст теоремы)
\end{theorem}

\begin{lemma}
kek=lol
\end{lemma}
\part*{Исторические сведения}
Возникновение теории вероятностей как науки относят к средним векам,
когда появилась возможность и возникла необходимость изучения математи-
ческими методами азартных игр (таких как орлянка, кости, рулетка). Самые
ранние работы учёных в области теории вероятностей относятся к XVII ве-
ку. Первоначально её основные понятия не имели строго математического
описания. Задачи, из которых позже выросла теория вероятностей представ-
ляли набор некоторых эмпирических фактов о свойствах реальных событий,
которые формулировались с помощью наглядных описаний. Исследуя про-
гнозирование выигрыша при бросании костей в письмах друг другу, Блез
Паскаль и Пьер Ферма открыли первые вероятностные закономерности. Ре-
шением тех же задач занимался и Христиан Гюйгенс. При этом с перепиской
Паскаля и Ферма он знаком не был и методику решения изобрёл самосто-
ятельно. Его статья, в которой он ввёл основные понятия теории вероятно-
стей (понятие вероятности как величину шанса; математическое ожидание
для дискретных случаев в виде цены шанса). В своей статье он использует
(не сформулированные ещё в явном виде) теоремы сложения и умножения
вероятностей. Статья была опубликована в печатном виде на двадцать лет
раньше (1657 г.) издания писем Паскаля и Ферма (1679 г.).
Важный вклад в теорию вероятностей внёс Якоб Бернулли, он дал до-
казательство закона больших чисел в простейшем случае независимых ис-
пытаний. В первой половине XIX века теория вероятностей начинает приме-
няться к анализу ошибок наблюдений; Лаплас и Пуассон доказали первые
предельные теоремы. Во второй половине XIX века основной вклад внесли
русские учёные П. Л. Чебышёв, А. А. Марков и А. М. Ляпунов. В это вре-
мя были доказаны закон больших чисел, центральная предельная теорема,
а также разработана теория цепей Маркова. Современный вид теория веро-
ятностей получила благодаря аксиоматике, предложенной Андреем Никола-
евичем Колмогоровым. В результате теория вероятностей приобрела строгий
математический вид и окончательно стала восприниматься как один из раз-
делов математики.

Википедия,
Статья "Теория вероятностей".

\part{События и их вероятности}
\section{Элементы комбинаторики. Схемы шансов}

В этом параграфе мы подсчитываем число элементарных событий или,
проще говоря, исходов, шансев, которые могут возникать в результате эксперимента. Например, при подбрасывании монеты могут произойти 2 исхода,
при подбрасывании игрального кубика могут произойти 6 исходов, при извлечении карты из колоды в 54 листа могут произойти 53 исхода. Такие подсчёты изучают в разделе математики, называемом комбинаторикой.
Пусть $A$ и $B$ — два непересекающихся конечных множества с числом
элементов $m$ и $n$ соответственно. Очевидны следующие две леммы.



\begin{lemma}(о сумме). Число шансов выбрать один элемент либо из $A$
либо из $B$, т.е. из объединения $A\cup B$, равно $m+n$.
\end{lemma}
\begin{lemma}(о произведении). Число шансов выбрать пару элементов,
один из $A$, а другой из $B$, равно $mn$, т.е. числу элементов в декартовом
произведении $A\times B$.
\end{lemma}
Непосредственным обобщением предыдущей леммы является следующая
теорема.
\begin{theorem}
Пусть $A_2,A_2,\dots,A_k$ — конечные непересекающиеся множества, имеющие $n_1
,n_2, \dots, n_k$ элементов соответственно. Выберем из
каждого множества по одному элементу. Тогда общее число способов, которыми можно осуществить такой выбор, равно $n_1n_2\dots n_k$.
\end{theorem}

Доказательство. 

Ясно, что число способов такого выбора равно числу точек (элементов) в декартовом произведении $A_1\times A_2\times\dots A_k$, т.е. равно $
n_1\cdot n_2\dots n_k$.
\subsection{Эксперименты выбора шариков}
Рассмотрим ящик, содержащий $n$ одинаковых шариков, на которых написаны
числа $1, 2,\dots, n$. Эксперимент состоит в том, что из ящика, не глядя, по
одному вынимают $k$ шариков, где $k ≤ n$. Обозначим через
(n1, n2, . . . , nk)
упорядоченный набор чисел, где n1 — номер 1-го вынутого шарика, n2 —
номер 2-го шарика, . . . , nk — номер k-го шарика.
Например, из 5 занумерованных шариков выбрали 3 шарика и получился
набор (4, 2, 1).

Сколько имеется различных способов вынуть из ящика k шариков? На
этот вопрос нельзя дать однозначный ответ, потому что такой эксперимент
определён неоднозначно.
Во-первых, не определено, возвращают ли извлеченный шарик обратно в
ящик. Во-вторых, не определено, какие наборы номеров считать различными
и какие наборы считать одинаковыми.
Рассмотрим следующие возможные условия проведения эксперимента.
1. Эксперимент с возвращением. Каждый извлечённый шарик возвраща-
ется в ящик.
В этом случае в наборе могут появляться одинаковые номера. Напри-
мер, при выборе трёх шариков из ящика, содержащего пять шариков
с номерами 1, 2, 3, 4 и 5, могут появиться наборы (3, 3, 5), (1, 2, 4)
и (4, 2, 1).
2. Эксперимент без возвращений. Извлечённые шарики в ящик не воз-
вращаются.
В этом случае в наборе не могут встречаться одинаковые номера. В
рассмотренном выше примере набор (3, 3, 5) не может появиться, а
наборы (1, 2, 4) и (4, 2, 1) могут.
Опишем теперь, какие наборы номеров мы будем считать различными.
Существуют ровно две возможности.
1. Эксперимент с учётом порядка. Два набора номеров считаются раз-
личными, если они отличаются либо составом, либо порядком.
В рассмотренном выше примере все наборы (3, 3, 5), (1, 2, 4) и (4, 2, 1)
считаются различными.
2. Эксперимент без учёта порядка. Два набора номеров считаются раз-
личными, если они отличаются только составом.
В рассмотренном выше примере наборы (1, 2, 4) и (4, 2, 1) доставляют
одно и тот же элементарное событие, а набор (3, 3, 5) — другое.
Подсчитаем теперь, сколько получится различных исходов для каждого
из четырёх экспериментов. Заметим, что в литературе такие эксперименты
часто называют схемами выбора или схемами шансов. Схема шансов — это
условия (с возвратом или без, какие наборы различны и т.д.), при которых
проводится эксперимент.

\subsection{Схема шансов без возвращения и с учетом порядка}


\subsection{Схема шансов без возвращения и без учёта порядка}

\subsection{Схема шансов с возвращением и с учётом порядка}

\subsection{Схема шансов с возвращением и без учёта порядка}

\subsection{Элементы комбинаторики. Схемы шансов}


\section{События, операции над ними и $\sigma$-алгебры событий}

\section{Вероятность и её свойства}


\section{Способы задания и подсчёта вероятности}

\subsection{Экспериментальное нахождение вероятности}


\subsection{Вероятность на конечном пространстве}

\subsection{Классическая вероятность}

\subsection{Вероятность на счётном пространстве}

\subsection{Геометрическая вероятность}

\section{Независимые события}

\section{Условная вероятность}

\section{Формула полной вероятности и формулы Байеса}

\section{Биномиальное распределение}

\section{$k$-номинальное распределение}

\section{Гипергеометрическое распрделение}

\part{Теория случайных величин}

\section{Случайные величины}

\section{Абсолютно непрерывные случайные величины}

\section{Функции Хевисайда и Дирака}

\section{Функции одной случайной величины}

\section{Случайные векторы и их распределения}

\section{Функции от двух случайных величин}

\section{Математическое ожидание}

\section{Дисперсия}

\section{Числовые характеристики зависимости
случайных величин}


\part{Законы больших чисел}

\section{Неравенство Бьенеме–Чебышёва и
неравенство Маркова}

\section{Последовательности случайных величин}

\section{Законы больших чисел}

\section{Предельные теоремы для
биномиального распределения}

\section{Характеристические функции}

\section{Вычисление характеристических
функций}

\section{Центральная предельная теорема}

\section{Сферическое, $\xi^2$-распределение
и распределение Стьюдента}

\section{Цепи Маркова}









\end{document}